\chapter*{Curriculum Vitae}
\label{chap:curriculum}
\addcontentsline{toc}{chapter}{Curriculum Vitae}
\renewcommand{\arraystretch}{1.5}
\pagestyle{plain}
% \vspace{1.5cm}
\noindent
\begin{tabular*}{\textwidth}{p{0.2\textwidth}p{0.75\textwidth}}
\multicolumn{2}{l}{\large Personal Details}\\
\toprule
Name& Kai Husmann\\
Date of birth& 03.07.1985\\
Place of birth&Sulingen\\
\end{tabular*}

% \vspace{1.5cm}
\noindent
\begin{tabular*}{\textwidth}{p{0.2\textwidth}p{0.75\textwidth}}
\multicolumn{2}{l}{\large Education}\\
\toprule
Since 10/2016& Master Student (M.Sc.)\newline University of G�ttingen, Applied Statistics\\
04/2014--10/2017& Ph.D. Student\newline University of G�ttingen, Forest Sciences and Forest Ecology\\
10/2010--01/2013&Master of Science (M.Sc.)\newline University of G�ttingen, Forest Sciences and Forest Ecology with study focus on Forest Ecosystem Analysis and Information Processing\\
10/2007--09/2010&Bachelor of Science (B.Sc.) \newline University of G�ttingen, Forest Sciences and Forest Ecology\\
\end{tabular*}

% \vspace{1.5cm}

\noindent
\begin{tabular*}{\textwidth}{p{0.2\textwidth}p{0.75\textwidth}}
\multicolumn{2}{l}{\large Professional Experience}\\
\toprule
Since 01/2017&Researcher \newline Department of Forest Economics and Forest Management, B�sgen Institute, University of G�ttingen\\
02/2013--12/2016&Researcher \newline Department of Forest Growth, Section of Growth Modelling and Computer Science, Northwest German Forest Research Institute
\end{tabular*}

% \vspace{1.5cm}

\noindent
\begin{tabular*}{\textwidth}{p{1\textwidth}p{0.05\textwidth}}
	\multicolumn{2}{l}{\large Publications in Conference Proceedings and Books}\\
	\toprule
	
	\begin{enumerate}[]
		\item [{[1]}] Rumpf, S., \textbf{Husmann, K.}, D�bbeler, H. (2013): Entscheidungswerkzeuge zur Sicherung einer nachhaltigen Rohstoffversorgung f�r die stoffliche und energetische Verwertung. In Abschlussbericht - Bioenergie Regionen St�rken (BEST) - Teilprojekt Schwachholzpotenzial Wald (IO-H4). \textit{Bundesministerium f�r Bildung und Forschung}. Berlin.
		
		\item [{[2]}] ML (2014): Der Wald in Niedersachsen. Ergebnisse der Bundeswaldinventur. \textit{Nieders�chsisches Ministerium f�r Ern�hrung, Landwirtschaft und Verbraucherschutz}. Hannover.
		
		\item [{[3]}] \textbf{Husmann, K.} (2015): Holzvolumenaggregation in Buchenkronen - Einsch�tzung des nutzbaren Holzvolumens �ber Randomized Branch Sampling. In \textit{Sektion Ertragskunde - Tagungsband 2015}. Kammerforst. ISSN 1432-2609
		
		\item[{[4]}] Auer, V., Zscheile, M., Engler, B., Haller, P., Hartig, J., Wehnser, J., \textbf{Husmann, K.}, Erler, J., Thole, V., Schulz, T., Hesse, E., R�ther, N., and Himsel, A. (2016): BIOECONOMY CLUSTER: resource efficient creation of value from beech wood to bio-based building materials. \textit{World Conference on Timber Engineering.} Vienna. ISBN: 9783903024359.
		
		\item[{[5]}] \textbf{Husmann, K.}, Saborowski, J., Hapla, F. (2016): Ursachenanalyse der Ringsch�le bei Edelkastanie (Castanea sativa [Mill.]) in Rheinland-Pfalz. In \textit{Die Edelkastanie am Oberrhein - Aspekte ihrer �kologie, Nuztung und Gef�hrdung}, volume 74/15 of \textit{Mitteilungen aus der Forschungsanstalt f�r Wald�kologie und Forstwirtschaft}. Trippstadt. ISSN 1610-7705
		
		\item[{[6]}] \textbf{Husmann, K.}, Nagel, J., Spellmann, H. (2017): VP 1.6 Perspektiven einer zukunftssicheren Logistik angewandt auf die nat�rliche Rohstoffversorgung in der Clusterregion, Schlussbericht \textit{BioEconomy Cluster}. G�ttingen.
		
		\item[{[7]}] \textbf{Husmann, K.}, Hansen, J., M�hring, B. (2017): Optimization of Thinning Intensity as Decision Support System for Forest Management. In \textit{Conference Program of the Symposium on Systems Analysis in Forest Resources}. Suquamish, Washington.
		
	\end{enumerate}	
&\\
\end{tabular*}

% \vspace{1.5cm}

\noindent
\begin{tabular*}{\textwidth}{p{1\textwidth}p{0.05\textwidth}}
	\multicolumn{2}{l}{\large Publications in Scientific Journals}\\
	\toprule
\begin{enumerate}[]
\item[{[1]}] 	\textbf{Husmann, K.}, M�hring, B. (2017): Modelling the economically viable wood in the crown of European beech trees. \textit{Forest Policy and Economics}, 78, 67-77. doi: https://doi.org/10.1016/j.forpol.2017.01.009

\item[{[2]}] \textbf{Husmann, K.}, Rumpf, S., Nagel, J. (2017): Biomass functions and nutrient contents of European beech, oak, sycamore maple and ash and their meaning for the biomass supply chain. \textit{Journal of Cleaner Production}. In Press. doi: https://doi.org/10.1016/j.jclepro.2017.03.019

\item[{[3]}] \textbf{Husmann, K.}, Saborowski, J., Hapla, F. (2013): Ursachenanalyse der Ringsch�le bei Edelkastanie (Castanea sativa [Mill.]) in Rheinland-Pfalz. \textit{forstarchiv}, 84:107-118.
\end{enumerate}
	&\\
\end{tabular*}

\noindent
\begin{tabular*}{\textwidth}{p{1\textwidth}p{0.05\textwidth}}
	\multicolumn{2}{l}{\large Publications in Other Journals}\\
	\toprule
	
	\begin{enumerate}[]
		\item [{[1]}] \textbf{Husmann, K.}, Auer, V., Beitzen-Heineke, I., Bischoff, H., Fehrensen, W.-G., Fischer, C., Gilly, A., Pfl�ger-Grone, H., Nagel, J., Spellmann, H., and Zscheile, M. (2016): Mittelfristigem Anstieg folgt stetiger R�ckgang - Zustand und Entwicklung der Rohholzverf�gbarkeit in der buchenreichen Mitte Deutschlands. \textit{Holz-Zentralblatt}, 37, 899-901.
		
		\item [{[2]}] Veronika, A., \textbf{Husmann, K.}, Gilly, A., Zscheile, M., Blendermann, H., Beitzen-Heineke, I. , Bischoff, H., Fehrensen, W., Pfl�ger-Grone, H., Schlehahn, H., Nagel, J., Spellmann, H. (2017): Bewertung der Buchenholz-Bereitstellungslogistik - Forschungsprojekt verdeutlicht Vorteile branchen�bergreifender Zusammenarbeit in der Holzbereitstellungskette. \textit{Holz-Zentralblatt}, 35, 806-808.
	\end{enumerate}	
	&\\
\end{tabular*}
