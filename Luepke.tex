%%%%%%%%%%%%%%%%%%%%%%% file template.tex %%%%%%%%%%%%%%%%%%%%%%%%%
%
% This is a general template file for the LaTeX package SVJour3
% for Springer journals.          Springer Heidelberg 2010/09/16
%
% Copy it to a new file with a new name and use it as the basis
% for your article. Delete % signs as needed.
%
% This template includes a few options for different layouts and
% content for various journals. Please consult a previous issue of
% your journal as needed.
%
%%%%%%%%%%%%%%%%%%%%%%%%%%%%%%%%%%%%%%%%%%%%%%%%%%%%%%%%%%%%%%%%%%%
%
% First comes an example EPS file -- just ignore it and
% proceed on the \documentclass line
% your LaTeX will extract the file if required
%\begin{filecontents*}{example.eps}
%!PS-Adobe-3.0 EPSF-3.0
%%BoundingBox: 19 19 221 221
%%CreationDate: Mon Sep 29 1997
%%Creator: programmed by hand (JK)
%%EndComments
%gsave
%newpath
%  20 20 moveto
%  20 220 lineto
%  220 220 lineto
%  220 20 lineto
%closepath
%2 setlinewidth
%gsave
%  .4 setgray fill
%grestore
%stroke
%grestore
%\end{filecontents*}
%
%\RequirePackage{fix-cm}
%
\documentclass[referee]{svjour3}                     % onecolumn (standard format)
%\documentclass[smallcondensed]{svjour3}     % onecolumn (ditto)
%\documentclass[smallextended]{svjour3}       % onecolumn (second format)
%\documentclass[twocolumn]{svjour3}          % twocolumn
%
\smartqed  % flush right qed marks, e.g. at end of proof
%
\usepackage{graphicx}
%
 \usepackage{mathptmx}      % use Times fonts if available on your TeX system
%
% insert here the call for the packages your document requires
%\usepackage{latexsym}
% etc.
%\usepackage[ngerman]{babel}
\usepackage[utf8x]{inputenc}
\usepackage{natbib}
\usepackage{multibib}

\usepackage{color}
\usepackage{rotating}
\setlength{\rotFPtop}{0pt plus 1fil}
\usepackage[breaklinks=true,draft=true]{hyperref}
\urlstyle{same}
\usepackage{amsmath}
\usepackage{multirow}
\usepackage{booktabs}
% please place your own definitions here and don't use \def but
% \newcommand{}{}
%

% Insert the name of "your journal" with
 \journalname{Eur J Forest Res}
%
\begin{document}

\title{Combining double sampling for stratification and cluster sampling to a three-phase sampling design for continuous forest inventories}
%\subtitle{Inclusion of growth model based updates of terrestrial sample plots in continuous resource inventories}
%Do you have a subtitle?\\ If so, write it here}

\titlerunning{Combining double sampling for stratification and cluster sampling}        % if too long for running head

\author{Nikolas von Lüpke \and
        Joachim Saborowski
}

%\authorrunning{Short form of author list} % if too long for running head

\institute{Nikolas von Lüpke \and Joachim Saborowski \at
              Department of Ecoinformatics, Biometrics and Forest Growth, University of Göttingen, Büsgenweg 4, D-37077 Göttingen, Germany \\
              Tel.: +49-551-393460, Fax: +49-551-393465\\
              \email{nluepke@gwdg.de}
           \and
Joachim Saborowski \at
             Department of Ecosystem Modelling, University of Göttingen, Büsgenweg 4, D-37077 Göttingen, Germany        
}

\date{Received: date / Accepted: date}
% The correct dates will be entered by the editor


\maketitle

%%%%%%%%%%%%%%
%% Abstract %%
%%%%%%%%%%%%%%

\begin{abstract}
We extend the well-known double sampling for stratification sampling scheme by cluster sub-sampling to a three-phase design and present corresponding estimators based on the infinite population approach in the first phase. After stratification of the sample points (phase I), a second-phase sample is drawn independently among the first-phase points within each stratum. For phase III, clusters are formed of those phase II points and a sample of clusters is finally drawn without replacement. We used the forest planning units compartment and sub-district as clusters and moreover formed clusters with a heuristic for the \textit{vehicle routing problem}. The precision of the new estimator was compared to that achieved with classical double sampling for stratification in a case study. The results indicate that the expected increase of sampling errors caused by clustering cannot be compensated by the reduced inventory costs under the conditions given in the case study.

\keywords{continuous forest inventory \and double sampling for stratification \and cluster sampling  \and infinite population approach}

% \PACS{PACS code1 \and PACS code2 \and more}
% \subclass{MSC code1 \and MSC code2 \and more}
\end{abstract}

%%%%%%%%%%%%%%%%%%
%% Introduction %%
%%%%%%%%%%%%%%%%%%

\section{Introduction}
\label{sec:introduction}
Generating statistically sound information is the main challenge of forest inventories, even though they additionally have to fulfil other demands \citep{Lund_1998}. The information gained from an inventory is valuable not only in the planning process of the forest-authorities or -owners. Moreover, the reporting for international conventions (e.g. biodiversity or climate change) needs statistically sound information. 

An important constraint in the development of an adequate inventory design is (cost-) efficiency. It is desirable to either achieve the highest precision with predefined inventory costs or to minimise the inventory costs for achieving a predefined precision.

One method that has prooved its efficiency and practicability often in the last decades is double sampling for stratification (2st) \citep{Cochran_1977, Vries_1986, Schreuder_1993, Koehl_1994, Saerndal_2003, Gregoire_2008, Mandallaz_2008}. \citet{Saborowski_2010} recently studied this method for periodic inventories under the infinite population approach, and \citet{Scott_1994} extended it by sampling with partial replacement (SPR). Two phases can be distinguished in this procedure: In the first phase qualitative information is obtained from a large number of sampling units. Based on these qualitative data, a stratification of the first-phase units is carried out and a sub-sample of every stratum is inventoried. The second-phase points are usually assumed to be chosen by simple random sampling (SRS) without replacement. Whereas it is common to obtain the qualitative data using aerial images, the quantitative data are usually gained through terrestrial sampling. The aim of the stratification is to create 
strata with a low 
within-strata 
variability and a high between-strata variability. In comparison with other methods the costs of this design are usually low \citep{Brassel_2001,Saborowski_2010}. Nevertheless, a further reduction of costs is desirable.

Another common and potentially efficient method is cluster sampling (cl), where spatial clusters of sampling units are formed and a sub-sample of these clusters is inventoried \citep{Cochran_1977, Gregoire_2008}. Usually the clusters are designed in a way that makes it possible to execute the field work per cluster within one day \citep{Kleinn_1996}. Several shapes of clusters such as triangle, square or cross exist, their efficiency has been studied e.g. by \citet{Kleinn_1994,Kleinn_1996}. The aim of clustering is to achieve a high variability within and a low variability between clusters. In comparison to SRS this method may be advantageous because of reduced travelling costs. Although the clustering always leads to a loss in precision compared to SRS \citep{Cochran_1977}, it might be more efficient if the precision of SRS can be achieved by a larger number of population elements sampled at lower costs. Cluster sampling is usually applied in tropical forests, where sample plots are difficult to access, but 
also for example in the national forest inventories of Finland and Germany \citep{NFI_2009}. 

It was conjectured by foresters that spatial clustering of second-phase units of the Lower Saxony State Forest District Inventories might also be more efficient, because clusters of an appropriate size could already cover a sufficiently large amount of variability and consequently the current relation between travelling- and inventory-time was detrimental. Therefore, we aimed at clustering the sample points of the second phase spatially, even though a loss in precision must be taken into account. In this article we present a new three-phase estimator with cluster subsampling of second-phase units and its application in a case study carried out with data of the State Forest District Inventory in Lower Saxony, Germany. 
%%%%%%%%%%%%%%%
%% Estimator %%
%%%%%%%%%%%%%%%
\section{An estimator for cluster subsampling of double sampling for stratification}
%%-----------------%%
%% Double Sampling %%
%%-----------------%%
\subsection{Double sampling for stratification}
\label{section:estimator_2st}
In this two-phase sampling scheme, measurements only take place in a sub-sample ($n$) of all first-phase plots ($n'$). Using low-cost qualitative variables, the first-phase sample points are stratified into $L$ strata. Strata means of the target variable ($\bar{y}_h$) are then weighted with the proportion of first-phase sample points per stratum ($n'_h/n'=w_h$) and summarised to estimate the overall mean $\bar{Y}$ (Eq. \ref{eq:2st}) \citep{Cochran_1977}.
\begin{equation}
\hat{\bar{Y}}_{2st}=\sum^L_{h=1}w_h\frac{1}{n_h}\sum_{i=1}^{n_h}y_{hi}=\sum^L_{h=1}w_h\bar{y}_h
\label{eq:2st}
\end{equation}
An unbiased estimator for the variance of this estimator under the infinite population approach \citep[see e.g.][]{Mandallaz_2008} is given by 
\begin{equation}
\hat{V}\left(\hat{\bar{Y}}_{2st}\right)=\frac{1}{n'-1}\left(\sum^L_{h=1}\frac{n'_h-1}{n'}\frac{s_h^2}{\nu_h}+\sum^L_{h=1}w_h\left(\bar{y}_h-\hat{\bar{Y}}_{2st}\right)^2\right)
\label{eq:2st_var}
\end{equation}

\begin{equation}
s_h^2=\frac{1}{n_h-1}\sum^{n_h}_{i=1}\left(y_{hi}-\bar{y}_h\right)^2
\label{eq:2st_var_within_stratum}
\end{equation}
\citep{Saborowski_2010}. Here, $s_h^2$ is the estimated within-stratum variance of the target variable (Eq. \ref{eq:2st_var_within_stratum}), and $\nu_h=n_h/n'_h$ is the proportion of terrestrial sample points per stratum.
%%------------------%%
%% Cluster Sampling %%
%%------------------%%
\subsection{Cluster sampling}
In cluster sampling, where a population is split into $K$ clusters, $k$ of which are randomly sampled, two estimation approaches exist, (i) the unbiased estimator (unb) and (ii) the Ratio-to-Size estimator (RtS) \citep[][with $n=k$ and $N=K$]{Cochran_1977}.
\subsubsection{Unbiased estimator}
An unbiased estimator of the population total $Y$ in a finite population of size $M_0=\sum_{i=1}^K M_i$ is
\begin{equation}
 \hat{Y}_{unb}=\frac{K}{k}\sum_{i=1}^k y_i,
\label{eq:y_unb}
\end{equation}
 where $y_i$ is the total of the target variable in cluster $i$. The corresponding variance is given by Eq. \ref{eq:var_unb}, where $\bar{Y}_{(K)}=Y/K$ is the population mean per cluster unit \citep{Cochran_1977}.
\begin{equation}
 V\left(\hat{Y}_{unb}\right)=\frac{K^2}{k}\left(1-\frac{k}{K}\right)\frac{\sum_{i=1}^K\left(y_i-\bar{Y}_{(K)}\right)^2}{K-1}
\label{eq:var_unb}
\end{equation}
The population mean per element ($\bar{\bar{Y}}$) can then be estimated by dividing the estimator of the population total by the total number of elements in the population ($M_0$) (Eq. \ref{eq:y_quer_unb}).
\begin{equation}
\hat{\bar{\bar{Y}}}_{unb}=\frac{\hat{Y}_{unb}}{M_0}=\frac{K}{kM_0}\sum_{i=1}^k y_i
\label{eq:y_quer_unb}
\end{equation}
\subsubsection{Ratio-to-Size estimator}
In this approach the population mean is estimated by the ratio of the sum of the target variables ($y_i$) in the sample to the total number of elements in the sample. 
\begin{equation}
\hat{\bar{\bar{Y}}}_{RtS}=\frac{\hat{Y}_{RtS}}{M_0}=\frac{\sum_{i=1}^k y_i}{\sum_{i=1}^k M_i}
\label{eq:y_quer_rtos}
\end{equation}
what is known to be an approximately unbiased estimator. Multiplying by the total number of elements in the population yields
\begin{equation}
 \hat{Y}_{RtS}=M_0\frac{\sum_{i=1}^k y_i}{\sum_{i=1}^k M_i},
\label{eq:y_rtos}
\end{equation}
an estimator for the population total. The formula of the corresponding approximate variance is similar to Eq. \ref{eq:var_unb}, the variance of the unbiased estimator. It differs only in that it replaces the mean per cluster $\bar{Y}_{(K)}$ by $M_i \bar{\bar{Y}}$.
\begin{equation}
 V\left(\hat{Y}_{RtS}\right)\doteq\frac{K^2}{k}\left(1-\frac{k}{K}\right)\frac{\sum_{i=1}^K\left(y_i-M_i\bar{\bar{Y}}\right)^2}{K-1} 
\label{eq:var_rtos}
\end{equation}
%%------------------------%%
%% Cluster subsampling of %%
%% double sampling        %%
%%------------------------%%
\subsection{Cluster subsampling of double sampling for stratification}
The envisaged procedure comprises three phases (Fig. \ref{fig:figure_1}). Phases I and II follow the well-known 2st design (see ``\nameref{section:estimator_2st}''). In the third phase, the second-phase units are clustered into $K$ clusters and $k \leq K$ clusters are randomly sampled without replacement. These $k$ sample clusters are finally measured in the field. Due to the fact that the Ratio-to-Size estimator mostly performed better than the unbiased estimator in our case study, we restrict the following presentation to the estimators and the results based on \eqref{eq:y_quer_rtos} and \eqref{eq:var_rtos}.
\begin{figure*}
  \includegraphics[width=\textwidth]{figure_1.pdf}
\caption{Sampling procedure of the three-phase sampling design.}
\label{fig:figure_1}
\end{figure*}
The estimator of the overall mean 
\begin{equation}
\hat{\bar{Y}}_{2st,cl}=\sum_{h=1}^{L}w_h\hat{\bar{y}}_h=\sum_{h=1}^{L}w_h\frac{\sum_{i=1}^k \breve{y}_{ih}}{\sum_{i=1}^k M_{ih}}
\label{eq:est_mean_rtos}
\end{equation}
differs from the 2st-estimator (Eq. \ref{eq:2st}) merely in the estimator of the mean per stratum ($\bar{y}_h$), which is now replaced by 
\begin{equation}
\hat{\bar{y}}_h=\frac{\sum_{i=1}^k \breve{y}_{ih}}{\sum_{i=1}^k M_{ih}}.
\label{eq:est_mean_strat_rts}
\end{equation}
$\breve{y}_{ih}=\sum_{l=1}^{M_{ih}}\breve{y}_{ihl}$ is the sum of the target variable over all second-phase units in cluster $i$ which belong to stratum $h$ (clusters may overlap different strata), and $M_{ih}$ is the number of those second-phase units. The estimator $\hat{\bar{y}}_h$ is nothing else than the Ratio-to-Size estimator for the mean per element (Eq. \ref{eq:y_quer_rtos}) in stratum $h$, if we consider the second-phase units as a finite population of size $M_0=n_h$. Thus \eqref{eq:est_mean_rtos} is also approximately unbiased.
%\begin{equation}
% \hat{\bar{y}}_h=\frac{\sum_{i=1}^k \breve{y}_{ih}}{\sum_{i=1}^k N_{ih}}
% \label{eq:rtos}
%\end{equation}

An estimator for the variance (Eq. \ref{eq:var_2stcl}) is given in Eq. \ref{eq:var_gesch_2stcl} (for the proof see the Appendix). 
\begin{equation}
\begin{aligned}
V\left(\hat{\bar{Y}}_{2st,cl}\right)&=\frac{1}{n'}\left(\sum_{h=1}^{L}\frac{W_h S^2_{h}}{\nu_h}+\sum_{h=1}^{L}W_h(\bar{Y}_h-\bar{Y})^2\right)\\
&+ E\sum_{h=1}^{L}\left(\frac{n'_h}{n'}\right)^2 \frac{1}{n_h^2}\frac{K^2}{k}\left(1-\frac{k}{K}\right)\breve{S}_h^2\\
&+E\sum_{h \neq h'}^{L}\frac{n'_h n'_{h'}}{n'^2}\frac{1}{n_h n_{h'}}\frac{K^2}{k}\left(1-\frac{k}{K}\right)\breve{S}_{hh'}.
\label{eq:var_2stcl}
\end{aligned}
\end{equation}

\begin{equation}
\begin{aligned}
\hat{V}\left(\hat{\bar{Y}}_{2st,cl}\right)&=\frac{1}{n'-1}\left(\sum_{h=1}^{L}\frac{n'_h-1}{n'}\frac{s^2_{h,cl}}{\nu_h}+\sum_{h=1}^{L}\frac{n'_h}{n'}(\hat{\bar{y}}_h-\hat{\bar{Y}}_{2st,cl})^2\right)\\
&+ \sum_{h=1}^{L}\left(\frac{n'_h}{n'}\right)^2 \frac{1}{n_h^2}\frac{K^2}{k}\left(1-\frac{k}{K}\right)\breve{s}_h^2\\
&+\frac{2K^2}{kn'^2}\left(1-\frac{k}{K}\right)\sum_{h<h'}^{L}\frac{n'_h n'_{h'}}{n_h n_{h'}}\breve{s}_{h h'}.    
\label{eq:var_gesch_2stcl}
\end{aligned}
\end{equation}
The first of the three main terms of \eqref{eq:var_2stcl} is the variance under 2st; terms for the variances within strata and the covariances between strata are added. 

Eq. \ref{eq:var_straten_2stcl} is the within-stratum variance between clusters, the corresponding estimator is given in Eq. \ref{eq:var_straten_gesch_2stcl}.
\begin{equation}
 \breve{S}_h^2=\frac{1}{K-1}\sum_{i=1}^K\left(\breve{y}_{ih}-M_{ih}\bar{y}_h\right)^2
\label{eq:var_straten_2stcl}
\end{equation}

\begin{equation}
 \breve{s}_h^2=\frac{1}{k-1}\sum_{i=1}^k\left(\breve{y}_{ih}-M_{ih}\hat{\bar{y}}_h\right)^2
\label{eq:var_straten_gesch_2stcl}
\end{equation}
$s_{h,cl}^2$, a conditionally unbiased estimator for $s_h^2$, given the first two phases, is given in Eq. \ref{eq:var_straten_gesch_2stcl_bedingt}, where $\bar{\breve{y}}_{ih}$ is the mean over all $M_{ih}$ sampling units in cluster $i$ and stratum $h$.
\begin{equation}
s_{h,cl}^2=\frac{1}{n_h-1}\frac{K}{k}\left[\sum_{i=1}^k M_{ih}\left(\bar{\breve{y}}_{ih}-\hat{\bar{y}}_h\right)^2+\sum_{i=1}^k\sum_{l=1}^{M_{ih}}\left(\breve{y}_{ihl}-\bar{\breve{y}}_{ih}\right)^2\right]
\label{eq:var_straten_gesch_2stcl_bedingt}
\end{equation}
Thus $s_{h,cl}^2$ is also unbiased for $S_{h}^2$. The covariance between the sampling units of different strata within a cluster can be calculated using Eq. \ref{eq:covar_straten_2stcl}, an estimator is given in Eq. \ref{eq:covar_straten_gesch_2stcl}.
\begin{equation}
 \breve{S}_{hh'}=\frac{1}{K-1}\sum_{i=1}^K\left(\breve{y}_{ih}-M_{ih}\bar{y}_h\right)\left(\breve{y}_{ih'}-M_{ih'}\bar{y}_{h'}\right)
\label{eq:covar_straten_2stcl}
\end{equation}

\begin{equation}
 \breve{s}_{hh'}=\frac{1}{k-1}\sum_{i=1}^K\left(\breve{y}_{ih}-M_{ih}\hat{\bar{y}}_h\right)\left(\breve{y}_{ih'}-M_{ih'}\hat{\bar{y}}_{h'}\right)
\label{eq:covar_straten_gesch_2stcl}
\end{equation}
The relative sampling error (rel. SE), as given in Eq. \ref{eq:rel_SE}, will also be interesting for evaluating the performance of the estimator. 
\begin{equation}
rel. SE=\frac{\sqrt{Var\hat{\bar{Y}}_{2st,cl}}}{\hat{\bar{Y}}_{2st,cl}}
\label{eq:rel_SE}
\end{equation}

%%%%%%%%%%%%%%%%
%% Case Study %%
%%%%%%%%%%%%%%%%
\section{Case Study}
%%-----------%%
%% Data Base %%
%%-----------%%
\subsection{Data Base} 
Since 1999 the State Forest District Inventory of Lower Saxony is carried out according to a 2st-design \citep{Boeckmann_1998}. In the first phase CIR aerial images are taken at every grid point of a 100 m $\times$ 100 m grid. Every sample point is assigned to one of eight strata, defined by age class (1: $\leq40$ years, 2: $> 40 - 80$ years, 3: $> 80 - 120$ years, 4: $ > 120$ years) and dominating species group (CON: Coniferous, DEC: Deciduous). Afterwards terrestrial sampling is carried out in a sub-sample of each stratum. The proportion of terrestrial sample points ($\nu_h$) differs between the strata. In every stratum the $n_h$ second-phase sample points are chosen systematically from a list of all sample points in the stratum; in the following study, we deal with the according sample plots as randomly selected plots in both phases. Measurements then are executed in two concentric circular plots. In a plot of radius 6 m all trees with a dbh $\geq 7$ cm and $< 30$ cm are measured. Trees with larger dbh 
are measured in 
a plot 
of 13 m radius. Tree heights were only partly measured during the inventory. In cases without height measurements we used species-specific height curves for estimation of tree heights \citep{Sloboda_1993}. Finally, the solid tree volumes were calculated with species-specific form factors \citep{Bergel_1973,Bergel_1974}.
The first run of the inventory has been carried out between 1999 and 2008 subsequently in all forest districts. Here, we used the data of seven state forest districts in the regions Harz (Clausthal, Lauterberg, Riefensbeek, Seesen) and Solling (Dassel, Neuhaus, Winnefeld) (Table \ref{tab:table_1}). Meanwhile forest districts were merged to larger units (see www.landesforsten.de). 
%%----------------------%%
%% Evaluation Procedure %%
%%----------------------%%
\subsection{Evaluation procedure}
The aim of the case study is to assess (i) the general performance of the 2st,cl-estimator compared to simple 2st and (ii) the effect of different cluster types on the variance of the estimator. As clusters we used (a) forest sub-districts (Revier), (b) compartments (Abteilung) and (c) daily workloads (Table \ref{tab:table_2}), the latter calculated with a heuristic for the vehicle routing problem (VRP) \citep{Dantzig_1959}, the Record-to-Record algorithm from \citet{Li_2005}. 

Very famous in Operations Research, the VRP describes the problem of supplying several customers from one depot with a truck of given capacity. It is desired to fulfil all customer demands, which can be different, and to find the shortest route. Adapting the algorithm to our problem, we assumed a daily working time of eight hours as capacity. Assuming as demand per sample plot a mean working time of 1.5 h for a two-people inventory-team \citep{Zinggeler_1997,Zinggeler_2001}, it is possible to measure five plots during a day on average. Due to the high number of sample 
plots per forest district and the lack of realistic depots, we used the forest sub-districts as unit for the calculations in case of this cluster type and assumed the starting point to lie in the center of all terrestrial sample plots of a sub-district. The clusters have been post-optimised with a 2-opt-algorithm (R library ``TSP") \citep{Hahsler_2007}. The workload-clusters were calculated with the VRPH-library \citep{Groeer_2010}, all other calculations were carried out with the statistical software package R \citep{R_Doc}.

Due to the fact, that data from all terrestrial sample points (phase II) and thus all clusters $K$ are available in the data sets, we used $\breve{S}_h^2$ (Eq. \ref{eq:var_straten_2stcl}) instead of $\breve{s}_h^2$ (Eq. \ref{eq:var_straten_gesch_2stcl}) and $s_{h}^2$ (Eq. \ref{eq:2st_var_within_stratum}) instead of $s_{h,cl}^2$ (Eq. \ref{eq:var_straten_gesch_2stcl_bedingt}) in the variance estimator (Eq. \ref{eq:var_gesch_2stcl}) to increase precision of the estimation and to avoid the otherwise necessary simulations of cluster sampling. For the same reasons, we replaced $\hat{\bar{y}}_h$ by $\bar{y}_h$ and $\hat{\bar{Y}}_{2st,cl}$ by $\hat{\bar{Y}}_{2st}$ and achieved
\begin{equation}
\begin{aligned}
\hat{V}\left(\hat{\bar{Y}}_{2st,cl}\right)&=\frac{1}{n'-1}\left(\sum_{h=1}^{L}\frac{n'_h-1}{n'}\frac{s^2_{h}}{\nu_h}+\sum_{h=1}^{L}\frac{n'_h}{n'}\left(\bar{y}_h-\hat{\bar{Y}}_{2st}\right)^2\right)\\
&+ \sum_{h=1}^{L}\left(\frac{n'_h}{n'}\right)^2 \frac{1}{n_h^2}\frac{K^2}{k}\left(1-\frac{k}{K}\right)\breve{S}_h^2\\
&+\frac{2K^2}{kn'^2}\left(1-\frac{k}{K}\right)\sum_{h<h'}^{L}\frac{n'_h n'_{h'}}{n_h n_{h'}}\breve{S}_{h h'}  
\label{eq:var_gesch_2stcl_ver}
\end{aligned}
\end{equation}
as an estimator for \eqref{eq:var_2stcl}. The first term equals the 2st variance estimator \eqref{eq:2st_var}. Please note that this estimator is not applicable in forest inventory practice because of the aforementioned substitutions. Finally, variances according to \eqref{eq:var_gesch_2stcl_ver} were calculated for three different target populations defined by species group and dbh: Beech $>$50 cm, Oak $>$50 cm, Spruce $>$35 cm.
\begin{table*}
\begin{center}
\tiny{
\caption{First- and second-phase sample sizes in the eight strata of the seven forest districts.}
\label{tab:table_1}
\begin{tabular}{lccrrrrrrrrr}
\hline\noalign{\smallskip}
 Forest \linebreak District & year & Phase & DEC1 & DEC2 & DEC3 & DEC4 & CON1 & CON2 & CON3 & CON4 & $\sum$\\
\noalign{\smallskip}\hline\noalign{\smallskip}

\multirow{2}{*}{Clausthal} &\multirow{2}{*}{1999, 2002} &  I  &  287 & 490 &  362 &  428 & 1445 &  5084 &  2250 &  1244 & 11590\\
		     && II  &  31 &  52 &  50 &  61 &  171 &  855 &  346 &   198 & 1764\\
\hline\noalign{\smallskip}
\multirow{2}{*}{Dassel} &\multirow{2}{*}{2000, 2001} &  I  &  1049 & 1205 &  1327 &  1613 & 1334 &  1928 &  895 &  188 & 9539\\
		     && II  &  145 &  132 &  159 &  280 &  281 &  578 &  241 &   52 & 1868\\
\hline\noalign{\smallskip}
\multirow{2}{*}{Lauterberg} &\multirow{2}{*}{2002} &  I  &  945 & 1269 &  1139 &  2091 & 1833 &  3783 &  2320 &  897 & 14277\\
		     && II  &  89 &  87 &  95 &  267 &  321 &  970 &  528 &   200 & 2557\\
\hline\noalign{\smallskip}
\multirow{2}{*}{Neuhaus} &\multirow{2}{*}{1999, 2000} &  I  &  1687 & 1521 &  1113 &  1663 & 1919 &  2326 &  1627 &  241 & 12097\\
		     && II  &  172 &  102 &  103 &  218 &  343 &  614 &  384 &   60 & 1996\\
\hline\noalign{\smallskip}
\multirow{2}{*}{Riefensbeek} &\multirow{2}{*}{2002} &  I  &  702 & 888 &  428 &  1014 & 1911 &  4971 &  1939 &  1178 & 13031\\
		     && II  &  68 &  54 &  37 &  124 &  293 &  1108 &  390 &   259 & 2333\\
\hline\noalign{\smallskip}
\multirow{2}{*}{Seesen} &\multirow{2}{*}{1999, 2001, 2002} &  I  &  1227 & 1328 &  707 &  1428 & 1229 &  3329 &  1306 &  375 & 10929\\
		     && II  &  238 &  230 &  184 &  374 &  271 &  855 &  320 &   78 & 2550\\
\hline\noalign{\smallskip}
\multirow{2}{*}{Winnefeld} &\multirow{2}{*}{2000} &  I  &  1664 & 2312 & 1573 & 2889 & 1501 & 1656 & 707 &  165 & 12467\\
		     && II  &  207 &  196 &  172 &  463 & 337 &  544 & 209 &  50 & 2178\\ 
\noalign{\smallskip}\hline
\end{tabular}}
\end{center}
\end{table*}

\begin{table*}
\begin{center}
\caption{Number and mean size of clusters per forest district defined by different rules. The values in brackets indicate the standard deviations.}
\label{tab:table_2}
\small{
\begin{tabular}{lrrrrrrr}
\hline\noalign{\smallskip}
 Forest \linebreak District & \multicolumn{2}{c}{compartment} & \multicolumn{2}{c}{VRP-cluster} &  \multicolumn{2}{c}{sub-district}\\
\cmidrule(lr){2-3}\cmidrule(lr){4-5}\cmidrule(lr){6-7}
&number&size&number&size&number&size\\
\noalign{\smallskip}\hline\noalign{\smallskip}
Clausthal &  667  &2.6&  357&4.94&9 & 196\\
&&$(\pm 1.35)$&&$(\pm 0.33)$&& $(\pm 17.01)$\\
Dassel &  524  & 3.56 &378&4.94&8 &233.5\\
&&$(\pm 2.07)$&&$(\pm 0.41)$&& $(\pm 48.35)$\\
Lauterberg &  726  & 3.52& 516&4.96& 11& 232.45\\
&&$(\pm 1.86)$&&$(\pm 0.34)$&& $(\pm 69.07)$\\
Neuhaus &  517  & 3.86&402 &4.97&10 & 199.6\\
&&$(\pm 1.96)$&&$(\pm 0.25)$&& $(\pm 34.40)$\\
Riefensbeek &  720&  3.24& 472&4.94&11 &  212.09\\
&&$(\pm 1.61)$&&$(\pm 0.39)$&& $(\pm 48.15)$\\
Seesen &  586  & 4.35&514 &4.96&10 & 255\\
&&$(\pm 3.13)$&&$(\pm 0.28)$&& $(\pm 118.19)$\\
Winnefeld &  701 & 3.11&  438 &4.97&11 &198\\
&&$(\pm 1.66)$&&$(\pm 0.22)$&& $(\pm 29.07)$\\
\noalign{\smallskip}\hline
\end{tabular}}
\end{center}
\end{table*}
For estimation of the travelling distances of 2st and 2st,cl we carried out simulations. In 100 simulation runs, a random sample of a given percentage of sample points or clusters was drawn without replacement from all terrestrial sample plots of a forest district. During clustering we already calculated the travelling-distances within each cluster and so the distances of the selected clusters had to be summarised for estimation of the overall 2st,cl-travelling-distance within clusters. To estimate the travelling-distance of 2st, we built clusters of sample plots per sub-district with the same procedure as described above for the \textit{VRP-clusters}. Afterwards the total distance within clusters per forest district was calculated.
%%---------%%
%% Results %%
%%---------%%
\subsection{Results}
The volumes per ha and rel. SEs, calculated with the full second phase sample size i.e. using \eqref{eq:2st} and \eqref{eq:2st_var}, are shown in Table \ref{tab:table_3}. It becomes obvious that relative precisions of about 5 \% and below could be achieved for the Spruces and Beeches in almost all forest districts. Only for the Oaks, which are generally rarer, and the small Beech population in Clausthal we observe lower rel. precisions, mostly above 10 \% and up to 33 \%.

Using the planning units \textit{compartment} and \textit{sub-district} as clusters leads to cluster sizes, which are either too big or too small to be sampled within one day (Table \ref{tab:table_2}). In addition, their size is highly variable as the standard deviations show. In contrast, the \textit{VRP}-clusters contain on average five points and therefore fit better to a one day workload. Besides, the variation in cluster size is low.

Using 2st,cl the rel. SE increases with decreasing percentage of sampled clusters (Fig. \ref{fig:figure_2}). The rel. SEs are lowest for the Spruces and highest for the Oaks. Sampling 50 \% of the \textit{VRP-clusters} the rel. SE varies between 7.53 \% and 21.44 \% for the Beeches, between 14.28 \% and 61.25 \% for the Oaks and between 3.37 \% and 5.5 \% for the Spruces. A comparison of the different cluster-forms shows that the differences between the \textit{compartments} and the \textit{VRP-clusters} are very small, whereas using the \textit{sub-districts} as clusters leads to higher rel. SE than using the two other cluster-forms. 

Comparing the 2st,cl-results with those of 2st assuming equal numbers of terrestrial sample plots, shows that the rel. SEs of 2st,cl are always higher than those of 2st (Fig. \ref{fig:figure_3}). When sampling 50 \% of the \textit{VRP-clusters}, the rel. SE of the Beeches is at least 25.78 \% and at most 33.53 \% higher than the corresponding value of 2st which can be achieved with the same number of terrestrial sample plots. Sampling the same percentage of \textit{VRP-clusters} leads to increases of the rel. SE between 25.22 \% and 40.96 \% for the Oaks and between 25.31 \% and 34.82 \% for the Spruces. With decreasing number of sampled clusters the rel. increase of the rel. SE increases nearly linearly. The cluster-forms \textit{
compartment} and \textit{VRP-cluster} lead to similar results for all tree species. Using the \textit{sub-districts} as clusters, on the contrary leads to worse results and pronounced differences between the tree species. The highest differences were calculated for the Spruces and the lowest for the Beeches.

The additional sampling effort of 2st,cl compared to 2st is shown in Fig. \ref{fig:figure_4} as a relative difference. Obviously, the same rel. SE can be achieved with less terrestrial sample points in 2st than in 2st,cl. The rel. increase of the number of sample points lies between 37.51 \% and 44.85 \% for the Beeches, between 36.26 \% and 49.9 \% for the Oaks and between 40.68 \% and 46.7 \% for the Spruces, when using 50 \% of the \textit{VRP-clusters}. These rel. increases correspond to absolute increases between 306 and 577 sample plots for the Beeches, between 344 and 680 sample plots for the Oaks and between 389 and 640 sample plots for the Spruces. With decreasing percentage of sampled clusters the rel. increase of sample points increases with a concave shape. Again, the differences between the results of the two cluster-forms \textit{compartment} and \textit{VRP-cluster} are small for all species. The use of the sub-districts as clusters 
leads to considerably worse results, a much higher percentage of points has to be sampled additionally. Moreover, the differences between the different target populations become more pronounced with the sub-districts as clusters. The lowest efficiency in the case of sub-districts is achieved for the Spruces, the highest for the Beeches.

On the other hand, drawing samples from single terrestrial sample plots leads to longer total within-cluster travelling-distances per forest district than drawing samples from clusters of sample plots (Fig. \ref{fig:figure_5}). When sampling 50 \% of the sample plots or clusters, the difference between the resulting mean overall travelling-distances per forest district lies between 94.95 km (Dassel) and 146.79 km (Seesen).

\begin{table*}
\begin{center}
\caption{Estimated volumes ($m^3 ha^{-1}$) and rel. SE (\%) for the 3 target populations in the different Forest Districts using double sampling for stratification. Bold numbers indicate target precisions below 5 \%.}
\label{tab:table_3}
 \begin{tabular}{lrrrrrrr}
\hline
&\multicolumn{2}{c}{Beech$>$50}&\multicolumn{2}{c}{Oak$>$50}&\multicolumn{2}{c}{Spruce$>$35}\\
\cmidrule(lr){2-3}\cmidrule(lr){4-5}\cmidrule(lr){6-7}
 Forest District & $\hat{\bar{Y}}$ & rel. SE &  $\hat{\bar{Y}}$ & rel. SE &  $\hat{\bar{Y}}$ & rel. SE\\
\hline
Clausthal & 8.61 & 12.10 & 3.33 & 16.00 & 132.96 & \textbf{2.19}\\
Dassel & 45.81 & \textbf{4.79} & 6.61 & 14.21 & 78.87 & \textbf{2.92}\\
Lauterberg & 26.05 & 5.61 & 3.89 & 18.07 & 111.88 & \textbf{1.97}\\
Neuhaus & 38.20 & 5.23 & 13.26 & 12.93 & 100.36 & \textbf{2.40}\\
Riefensbeek & 15.25 & 7.87 & 1.93 & 33.37 & 145.64 & \textbf{1.90}\\
Seesen & 38.15 & \textbf{4.44} & 4.57 & 13.98 & 82.97 & \textbf{2.44}\\
Winnefeld & 58.98 & \textbf{4.06} & 25.26 & 7.52 & 55.63 & \textbf{3.28}\\
\hline
 \end{tabular}
\end{center}
\end{table*}

\begin{figure*}
  \includegraphics[width=\textwidth]{figure_2.pdf}
\caption{Rel. SE [\%], calculated with 2st,cl and the Ratio-to-Size approach, as a function of the percentage of sampled clusters. The results are shown for three different target populations and cluster-forms in seven forest districts. The dotted horizontal line depicts the 5 \% level usually strived for in inventories of bigger diameter classes. The broken lines depict Minimum and Maximum of the rel. SEs, when sampling 50 \% of the VRP-clusters.}
\label{fig:figure_2}
\end{figure*}


\begin{figure*}
  \includegraphics[width=\textwidth]{figure_3.pdf}
\caption{The differences between the rel. SEs of 2st,cl and 2st as percentage of the rel. SE of 2st in seven forest districts, assuming identical numbers of terrestrial sample plots for both designs. The broken lines depict Minimum and Maximum of the difference, when sampling 50 \% of the VRP-clusters.}
\label{fig:figure_3}
\end{figure*}

\begin{figure*}
  \includegraphics[width=\textwidth]{figure_4.pdf}
\caption{The relative increase of the number of sample plots (phase II) measured in 2st,cl compared to the number of sample plots which is necessary to achieve the same precision using 2st. The broken lines depict Minimum and Maximum of the relative increase, when sampling 50 \% of the VRP-clusters.}
\label{fig:figure_4}
\end{figure*}

\begin{figure*}
\begin{center}
  \includegraphics[height=0.9\textheight]{figure_5.pdf}
\caption{The estimated total travelling-distances per forest district. In 100 simulation runs 
a sample was randomly drawn without replacement from either single terrestrial sample plots (SRS) or clusters of sample plots (Cl). Clustering of the sample plots with a VRP-algorithm and calculation of the travelling-distance within a cluster were executed afterwards for the first case. For the second case these calculations were executed based on all terrestrial sample plots.}
\label{fig:figure_5}
\end{center}
\end{figure*}

%%------------%%
%% Discussion %%
%%------------%%
\subsection{Discussion and Conclusions}
Coming back to the initial question of the general performance of the 2st,cl-estimator, we quantified additional sampling efforts and vice versa losses in precision, compared to pure 2st, for three target populations in seven forest districts. The additional sampling efforts can be justified in cases where SRS leads to a higher time consumption between sampling points than 2st,cl. However, the results of our case study indicate that the additional travelling-costs for SRS must be extremely large for justifying the additional sampling effort through 2st,cl. Assuming a time consumption of 1.5 h per plot for a two-people inventory-team, like in the Second Swiss National Forest Inventory \citep{Zinggeler_1997,Zinggeler_2001}, an additional sampling-effort of 306 to 577 sample plots corresponds to an additional time-consumption of 459 h to 856.5 h for sampling, disregarding the additional time-consumption for travelling (Beech, 50 \% of the \textit{VRP-clusters}). In contrast, the distance-reduction through 
clustering of the sample plots lies between 94.95 km and 146.79 km for the same sampled perecentage 
and cluster-type. These distances correspond to time-consumptions of 28.51 h and 44.08 h, assuming a walking-speed of 3.33 km/h, as given by \citet{Scott_1993} for medium terrain. For longer distances this speed is a conservative estimation, it is likely that longer distances between sample plots will be covered by car and hence faster. Longer distances can especially be expected for 2st with reduced sample sizes. It has to be kept in mind that the considered distances only account for the distances within clusters; distances between starting point and sample points are disregarded. So, the resulting distances are only an approximation. In our opinion it is reasonable to focus on the within-cluster distances, because they are assumed to depend on the cluster-sizes, whereas the distances between starting point and clusters are independent from the cluster-sizes. Due to the fact that the same method has been applied for estimating the travel distances of the two methods, we assume the values to be comparable. 
Moreover, it is realistic to assume walking between sample plots.

Thus, for relatively small planning units with a good infrastructure, like the forest districts in our case study, 2st,cl cannot achieve a higher efficiency than 2st. However, in large areas with bad infrastructure and access to the points, travelling between sample points might be so time-consuming that 2st,cl can be advantageous.

The second aim of our study was to assess the effect of different cluster forms on the performance of the 2st,cl-estimator. Due to their highly variable and inadequate size, the use of the compartments and sub-districts as clusters is detrimental and \textit{VRP-clusters} thus preferable. Nevertheless, the resulting 2st,cl-variances of the \textit{compartment}- and \textit{VRP-clusters} are similar and high, compared to the values of pure 2st. This may be explained by their similar and small sizes, which most likely lead to low within-cluster variablility. The stand type at neighbouring sample points will often be similar, contradicting the aim of creating heterogenous clusters. Creating clusters being spatially compact and, at the same time, of high within-cluster variability can hardly be achieved under the conditions in this case study.
%%%%%%%%%%%%%%%%%%%%%%
%% Acknowledgements %%
%%%%%%%%%%%%%%%%%%%%%%
\begin{acknowledgements}
We thank the German Science Foundation (DFG) for financial support of this study (Sachbeihilfe SA 415/5-1) and Dr. Böckmann of the Lower Saxony Forest Planning Office for his kind provision of the inventory data.
\end{acknowledgements}

% BibTeX users please use one of
\bibliographystyle{spbasic}      % basic style, author-year citations
%\bibliographystyle{spmpsci}      % mathematics and physical sciences
%\bibliographystyle{spphys}       % APS-like style for physics

\bibliography{Luepke}


\appendix
\section{Appendix}
\label{sec:app}
\renewcommand{\theequation}{\thesection.\arabic{equation}} 
\renewcommand{\thetable}{\thesection.\arabic{table}} 
\setcounter{equation}{0}
\setcounter{table}{0}

\subsection{Proofs}
To derive the variance of the new estimator, we decompose $\hat{\bar{y}}_h$ as follows
\begin{equation}
 \hat{\bar{y}}_h=\bar{y}_h+\left(\hat{\bar{y}}_h-\bar{y}_h\right).
\end{equation}
The variance of $\hat{\bar{y}}_h$ is then given as the sum of the variances of the two components, because both components are not correlated \citep[see (12.6) in][]{Cochran_1977}.
\begin{equation}
 Var\hat{\bar{Y}}_{cl}=Var\sum_{h=1}^{L}\frac{n'_h}{n'}\bar{y}_h+Var\sum_{h=1}^{L}\frac{n'_h}{n'}\left(\hat{\bar{y}}_h-\bar{y}_h\right)
\end{equation}
In this equation the first variance is the variance from 2st (Eq. \ref{eq:2st_var}). Due to the fact that $E_3(\hat{\bar{y}}_h-\bar{y}_h)=0$,
\begin{equation}
 Var\sum_{h=1}^{L}\frac{n'_h}{n'}\left(\hat{\bar{y}}_h-\bar{y}_h\right)=E Var_3\sum_{h=1}^{L}\frac{n'_h}{n'}\left(\hat{\bar{y}}_h-\bar{y}_h\right)
\label{eq:app:var_diff}
\end{equation}
holds for the second variance. Assuming simple random sampling with drawing without replacement for the clusters, the variance and the covariance can be calculated as
\begin{equation}
Var_3\left(\hat{\bar{y}}_h-\bar{y}_h\right)=Var_3\hat{\bar{y}}_h=\frac{1}{n_h^2}\frac{K^2}{k}\left(1-\frac{k}{K}\right)\breve{S}_h^2
\label{eq:app:var3_diff}
\end{equation}
and
\begin{equation}
Cov_3 \left(\hat{\bar{y}}_h-\bar{y}_h,\hat{\bar{y}}_{h'}-\bar{y}_{h'}\right)=Cov_3\left(\hat{\bar{y}}_{h},\hat{\bar{y}}_{h'}\right)=\frac{1}{n_h n_{h'}}\frac{K^2}{k}\left(1-\frac{k}{K}\right)\breve{S}_{hh'}
\label{eq:app:covar3_diff}
\end{equation}
respectively. Substituting \eqref{eq:app:var3_diff} and \eqref{eq:app:covar3_diff} in \eqref{eq:app:var_diff} yields:
\begin{equation}
\begin{aligned}
 Var\sum_{h=1}^{L}\frac{n'_h}{n'}\left(\hat{\bar{y}}_h-\bar{y}_h\right)=&E\sum_{h=1}^{L}\left(\frac{n'_h}{n'}\right)^2 \frac{1}{n_h^2}\frac{K^2}{k}\left(1-\frac{k}{K}\right)\breve{S}_h^2\\
&+E\sum_{h \neq h'}^{L}\frac{n'_h n'_{h'}}{n'^2}\frac{1}{n_h n_{h'}}\frac{K^2}{k}\left(1-\frac{k}{K}\right)\breve{S}_{hh'}.
\end{aligned}
\end{equation}


\end{document}
% end of file template.tex

 


 


  






  
  

