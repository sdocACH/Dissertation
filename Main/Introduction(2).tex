%\setchapterpreamble[uc][0.8\textwidth]{%
%	\dictum[George Bernard Shaw]{%
%		``It is the mark of a truly intelligent person to be moved by statistics.''}\vskip20em}
	
\setchapterpreamble[uc][0.8\textwidth]{%
			\dictum[George Bernard Shaw]{%
				``Science never solves a problem without creating ten more.''}\vskip20em}

%\setchapterpreamble[uc][0.8\textwidth]{%
%	\dictum[Han Solo]{%
%		``She may not look like much, but she's got it where it counts.''}\vskip20em}			
	

	
\chapter{Introduction}
\label{chap:intro}
\newpage
%------------------------------------------------------------------------------------------------%
% ---------------------------- History and state of the art------------------------------------- %
\section{Short history of forest inventory}
\label{sec:intro:hist_soa}

%This first section has the objective to outline important developments in forest inventories that subsequently led to the recent state of methods to which this thesis was intended to contribute. 
The first known conductions of forest inventories date back to the 14th and 15th century. They looked quite different from today's inventories and exclusively comprised a visual inspection that was carried out by riding or walking through the forests. These inspections were a means to acquire a representative impression about the state of large forest areas as well as to determine spatial units for harvesting \citep{zoehrer1980}. Whereas these inspections constituted a response to already increasing wood shortages, the first reference to sustainable forest management in literature is only found two centuries later in the book \textit{Silviculatura Oeconomica} written by \citet{carlowitz1713}, who suggested concepts that besides reforestation mostly targeted at a 'continuous and sustainable' use of wood. However, these concepts required updated, profound and reliable information about the forested areas.\par

While the first use of sample plots to gather representative information about the state and development of forests date back to \citet{hartig1795}, major advancements of sample-based inventories came up at the beginning of the 20th century together with the development of statistical sampling methods. In North America, the first sample-based inventories (so-called \textit{timber cruises}) were conducted around 1930. The surveys were initially performed by visual assessments and later by a full census of all trees along systematically arranged lines. The idea of these so-called \textit{strip-sampling} inventories was to gather information only for a small percentage of the forest, and this information was then subsequently extrapolated to the entire forest area. In addition, the surveyed strips were also used to provide forest attribute maps. These strip sampling techniques were further developed by \citet{goodspeed1934} and \citet{langballe1938} who both proposed to collect information only within sample plots aligned with the strip lines (line-plot sampling) in substitution for a complete census within the stripes as a means to reduce costs and to make the cruises more efficient.\par 

In addition to improving the efficiency of sampling techniques, it also became of high importance to reliably quantify the estimation errors associated to estimated forest attributes. Solutions were developed in two mathematical frameworks of inference that rely on either randomization of the sampling process or sampling from an underlying stochastic process and are today known as the \textit{design-based} and \textit{model-dependent} approach. Randomization of the sample plot locations in order to allow for a valid estimate of the sampling error was first recommended by \citet{hasel1938}. A substantial advancement in design-based survey sampling constituted the concept of unequal probability sampling by \citet{hansen1943}, who showed that using inclusion probabilities proportional to the value of the target attribute (so-called \textit{probability proportional to size} or \textit{PPS} sampling) could substantially increase estimation precision. An unbiased estimator for unequal probability sampling was then contributed by \citet{horvitz1952}. The concept of \textit{PPS} sampling is implemented in most of today's forest inventories by the use of concentric sample plots. A method which perfectly implements PPS sampling is the angle-count sampling (ACS) technique introduced by Bitterlich in 1947 \citep{bitterlich1984}, and it was \citet{grosenbaugh1958} who related ACS to the probabilistic sampling theory. A further important development was the reformulation of the design-based estimation frame, in particular the Horwitz-Thompson estimator, within the \textit{infinite population approach} by \citet{mandallaz1991}, which provided a much better definition of the underlying population for forest inventories compared to design-based survey methods for finite populations as applied in official statistics \citep[e.g.]{sarndal2003}.\par

The huge advancements in the conduction of national forest inventories in the Nordic countries around 1920 considerably contributed to the development of estimators in the model-dependent framework, since these inventories used the concept of systematic strip sampling. Especially the variance estimators for systematic sampling by Mat\'{e}rn (1947, 1960) were used to quantify the estimation precisions \citep{kangas2006}. Even while Mat\'{e}rns variance estimators relied on modelling spatial trends, it was \citet{houllier1987} and then \citet{mandallaz1993} who first applied geospatial kriging techniques in the field of forest inventory.\par


%Since the 1950s, the demands on forest services have steadily been increasing. Beside the role of forests as wood supply, more diverse functions of forests became of importance, such as the recreational role, conservation of biodiversity, forest health and vitality (McRoberts et al. 2010) and also regulating and protective functions (Berger and Rey 2004; Frehner et al. 2005). As a result, today's forest inventories have to collect a huge amount of variables in order to provide quantitative data to report on each of these information needs. This has been supported by the development of a broad range of concepts and methods particularly suited to estimate forest inventory attributes \citep{gregoire2007, kohl2006, mandallaz2008, schreuder1993}. The large number of attributes to be collected has also further limited the number of sample plots than can be surveyed in a given amount of time, in particular under recent cost-restrictions. As a result, there has been an increasing need for alternative inventory methods that can maintain estimation precision at lower costs, i.e. fewer terrestrial sample sizes, or achieve higher estimation precision at identical costs \citep{vonluepke2013}. 
%
%A method which has been intensively used in this respect is multi-phase sampling. The core concept of such sampling procedures is to enlarge the sample size in order to gain higher estimation precision without enlarging the terrestrial sample size. This is done by using predictions of the target variable at additional sample locations where the terrestrial information has not been gathered. These predictions are generated by statistical models, commonly regression models, that use explanatory variables derived from auxiliary data. Regression estimators using this concept ususally use two samples of different sampling intensities (double- or two-phase sampling). According to \citet{schreuder1993}, double-sampling methods had already been developed by Cochran (1939), Neyman (1938) and Watson (1937) before they were extensively formulated in the forest inventory context by \citet{schreuder1993, kohl2006, gregoire2007, mandallaz2008}.
%
%Application of double-sampling methods in forestry became particularly interesting with the availablity of suitable auxiliary information. For a long time, this information was exclusively provided by aerial photographs. In Canada, aerial photographs had a long tradition and were already used in the 1930s in order to locate the inventory strips through merchantable forest stands. In Switzerland, Schmidt (1970) suggested a two-phase method where tree heights, stand densities, crown diameters and stem numbers are determined by a visual interpretation of aerial photos, and the terrestrial inventory is carried out only at a subsample. This method was then indeed implemented in the first Swiss NFI in 1983. Also Zöhrer (1980) in Germany already stated the use of areal photographs as a state-of-the-art of modern forest inventories. He described a comprehensive catalog of forest attributes to be derived from aerial photo interpretation, amongst others tree species, crown diameter, tree heigths as well as timber volume estimates on tree and stand level. He also suggested to use the resulting forest attribute maps for forest inventories in terms of building strata of different sampling frequencies.

% Major advancements in the incorporation of auxiliary data in forest inventory however emerged with a) the availability of large-scale remote sensing data technologies and b) increased computation capacities that allowed for automatically deriving information from that sources.

% Particularly the availability of small-footprint airborne laserscanning data 



% Recently, \citet{mandallaz2013c} and \citet{vonLuepke2012} extended double-sampling to triple-sampling and provided regression estimators that can use auxiliary information derived at two different sampling intensities.

% ...

%%This first section has the objective to outline important developments in forest inventories that subsequently led to the recent state of methods to which this thesis was intended to contribute. 
%The first known conductions of forest inventories date back to the 14th and 15th century. They looked quite different from today's inventories and exclusively comprised a visual inspection that was carried out by riding or walking through the forest. These inspections were a means to acquire a representative impression about the state of large forest areas as well as to determine spatial units for harvesting \citep{zoehrer1980}. The ability to estimate forest attributes by eye remained a crucial ability of forest practicioners for the next five centuries and is today still applied for stand inventories in many countries. Whereas these inspections constituted a response to increasing wood shortages, the first reference to sustainable forest management in literature is only found two centuries later in the book \textit{Silviculatura Oeconomica} written by \citet{carlowitz1713}. With the objective of a 'continuous and sustainable usage', Carlowitz suggested concepts that besides reforstation mostly targeted a more efficient use of wood.\par
%
%While the first use of sample plots to gather representative information date back to \citet{hartig1795}, major advancements of sample-based inventories came up at the beginning of the 20th century together with the development of statistical sampling methods. In North America, the first sample-based inventories (so-called \textit{timber cruises}) were conducted around 1930. The surveys were initially performed by visual assessment and later by a full census of all trees along systematically arranged lines. The idea of these so-called \textit{strip-sampling} inventories was to gather information only for a small percentage of the forest, which was then subsequently extrapolated to the entire forest area. The use of strips was at the same time also a means to produce forest attribute maps. An important further development of this sampling technique was the work of \citet{goodspeed1934} who advocated the collection of data within sample plots aligned with the strip lines (line-plot sampling) in substitution for a complete census within the stripes in order to improve the efficiency of the cruises. At the same time, \citet{langballe1938} demonstrated an early two-stage sampling approach in Canada in order to further reduce costs of a terrestrial cruise. They were confronted with considerable increases in survey costs triggered by stands becoming smaller, which required closer line spacing and more plots surveyed in the cruise. They thus proposed a method where field crews quickly counted the trees within each strip by eye, whereas they only performed time-consuming measurements at few plots within the strip. They subsequently used the relationship between the target variable (basal area) and the counted stem number to extrapolate the plot measurements to the entire strip area. This concept can be regarded as an early version of modern two-phase or two-stage estimation techniques.\par 
%
%Beside improving the efficiency, an issue that also became of high importance was to quantify the estimation error when sampling techniques were used. Solutions to this were developed in two mathematical frameworks that today constitute the two approaches for inference of estimates. One approach was the randomization of the plot locations, which was first recommended by \citet{hasel1938} in order to allow for a valid estimate of the sampling error under random sampling by applying the technique of variance analysis suggested by \citet{fisher1925}. For actual implementation, he suggested to devide the forest area into equally sized and shaped blocks and then randomly select a subset of these plots to be terrestrially surveyed. He thereby also described the so-called finite-population setup that builds the framework for a broad range of sampling methods \citep{schreuder1993}. At the same time, also other countries made huge advancement in the conduction of forest inventory methods. Around 1920, the first national, i.e. country-wide forest inventories (NFI) were conducted in the Nordic countries. Likewise the inventories in North America and Canada, also the Nordic NFIs were based on systematic strip sampling. This led to considerable contributions to variance estimation under \textit{systematic sampling}, particularly by Mat\'{e}rn (1947, 1960) who developed a variance estimator that relies on modeling spatial trends. The two theoretical frameworks of inference thus relied on either randomization of the sampling process or sampling from an underlying stochastic process and are today known as the \textit{design-based} and \textit{model-dependent} approach. A defining advancement in design-based survey sampling was made by the introduction of unequal propability sampling by \citet{hansen1943}, who showed that using inclusion probabilities proportional to the value of the target attribute could substantially increase estimation precision. An unbiased estimator for unequal probability sampling was contributed by \citet{horvitz1952}. The concept of \textit{probability proportional to size} (PPS) sampling is today implemented in most forest inventories by the use of concentric sample plots. A method which, even without intention at the time of invention, perfectly realizes PPS sampling is the angle-count sampling (ACS) technique introduced by Bitterlich in 1947 \citep{bitterlich1984}, and it was \citet{grosenbaugh1958} who later related ACS to the probabilistic sampling theory. A further important development was the reformulation of the design-based estimation frame within the \textit{infinite population approach} by \citet{mandallaz2008}. This approach provided a much better definition of the underlying population for forest inventories compared to design-based survey methods for finite populations as applied in official statistics.\par
%
%Since the 1950s, the demands on forest services have steadily been increasing. Beside the role of forests as wood supply, more diverse functions of forests became of importance, such as the recreational role, conservation of biodiversity, forest health and vitality (McRoberts et al. 2010) and also regulating and protective functions (Berger and Rey 2004; Frehner et al. 2005). As a result, today's forest inventories have to collect a huge amount of variables in order to provide quantitative data to report on each of these information needs. This has been supported by the development of a broad range of concepts and methods particularly suited to estimate forest inventory attributes \citep{gregoire2007, kohl2006, mandallaz2008, schreuder1993}. The large number of attributes to be collected has also further limited the number of sample plots than can be surveyed in a given amount of time, in particular under recent cost-restrictions. As a result, there has been an increasing need for alternative inventory methods that can maintain estimation precision at lower costs, i.e. fewer terrestrial sample sizes, or achieve higher estimation precision at identical costs \citep{vonluepke2013}. 
%
%A method which has been intensively used in this respect is multi-phase sampling. The core concept of such sampling procedures is to enlarge the sample size in order to gain higher estimation precision without enlarging the terrestrial sample size. This is done by using predictions of the target variable at additional sample locations where the terrestrial information has not been gathered. These predictions are generated by statistical models, commonly regression models, that use explanatory variables derived from auxiliary data. Regression estimators using this concept ususally use two samples of different sampling intensities (double- or two-phase sampling). According to \citet{schreuder1993}, double-sampling methods had already been developed by Cochran (1939), Neyman (1938) and Watson (1937) before they were extensively formulated in the forest inventory context by \citet{schreuder1993, kohl2006, gregoire2007, mandallaz2008}.
%
%Application of double-sampling methods in forestry became particularly interesting with the availablity of suitable auxiliary information. For a long time, this information was exclusively provided by aerial photographs. In Canada, aerial photographs had a long tradition and were already used in the 1930s in order to locate the inventory strips through merchantable forest stands. In Switzerland, Schmidt (1970) suggested a two-phase method where tree heights, stand densities, crown diameters and stem numbers are determined by a visual interpretation of aerial photos, and the terrestrial inventory is carried out only at a subsample. This method was then indeed implemented in the first Swiss NFI in 1983. Also Zöhrer (1980) in Germany already stated the use of areal photographs as a state-of-the-art of modern forest inventories. He described a comprehensive catalog of forest attributes to be derived from aerial photo interpretation, amongst others tree species, crown diameter, tree heigths as well as timber volume estimates on tree and stand level. He also suggested to use the resulting forest attribute maps for forest inventories in terms of building strata of different sampling frequencies.






% end with: A conclusion from recent developments is that future forest planning methods will increasingly integrate information of auxiliary data such as remote sensing and geo data. 

\newpage
%------------------------------------------------------------------------------------------------%
% ---------------------------------- Goals ----------------------------------------------------- %
\section{Thesis objective and structure}
\label{sec:intro:obj_and_struct}


% focus on strategy:
% - evaluate the tested methods with respect to their operational use, i.e. become an integral part of future forest inventories.
% Criteria are:
%      -> operationally applicable over large forest areas (country, - state-wide)
%      -> transferability to areas of application
%      -> effort to frequently produce estimates of numerous forest attributes 
%      -> provide reliable accuracy specifications
%
% -> this also included to add / develop / suggest techniques to process the auxiliary data
% -> particularly including dealing with quality restrictions in the auxiliary data 


The objective of this thesis was to contribute methodologies to the recent developments in combining existing forest inventory data from field surveys with auxiliary data derived from remote sensing data. The particular focus was to investigate the potential increase in value of already existing large scale terrestrial forest inventories to be used on small-scale management levels. The main question of this thesis to be addressed was: what estimation accuracies can be achieved using forest inventory data on much smaller spatial levels as their sampling intensities have originally been designed for? This question was investigated in two main studies: The first study \hyperref[sec:study1]{(study 1)} constitutes the main part of this thesis and concentrated on exploring the capabilities and performances of design-based multi-phase regression estimators in the service of small area estimation. The second study \hyperref[sec:study2]{(study 2)} investigated a new approach for evaluating the estimation accuracies of forest attribute maps, which are considered to be a special case of small area estimation. The following subsections will give a more detailed introduction to each of the studies.


% --------------------------------- %
% --------------------------------- %
\subsection{Study 1: Design-based small area estimation}
\label{sec:study1}

% ############################################################
% ## ===== general information on study 1 (RLP SAEs) ====== ##

This study constituted the major work of the thesis and had the objective to develop and evaluate a double-sampling estimation procedure for the German National Forest Inventory (German NFI). The particular objective of the study was to investigate whether the use of the German NFI data can provide acceptable estimation precision on two forest district levels when incorporated in small area estimation procedures. Similar studies have been conducted in Norway \citep{breidenbach2012} and Switzerland \citep{magnussen2014a, steinmann2013}, but no extensive study had yet been available for Germany. The results from this study were considered to provide valuable evidence whether a double-sampling extension of the German NFI might be a cost-saving alternative to a regional terrestrial forest district inventory (FDI). For this reason, it was a prerequisite to gather information over a sufficiently large number of small area units in order to allow for reliable conclusions. The study was conducted in the German federal state Rhineland-Palatinate (RLP) where the German NFI was extended to a double-sampling design and three types of small area regression estimators were applied in order to derive point and variance estimates of mean standing timber volume on two forest district levels comprising 45 and 405 units respectively.\par

This thesis focussed on exploring the performances of design-based regression estimators in the infinite population approach. Methods for this family of estimators have considerably been contributed to by the works of \citet{mandallaz2008, mandallaz2013a, mandallaz2013c} and \citet{mandallaz2013b}. The applicability of these estimators to existing national forest inventories for global estimation has recently been investigated by \citet{massey2015_thesis}. The work on this thesis was also a continuation in the application of these estimators for the special case of small area estimation.

Consequently, this thesis continued the exploration of these estimators for the special case of small area estimation.


 The design-based double-sampling estimators suggested by Mandallaz were also favored for the following reasons: Firstly, the estimators are explicitly formulated for cluster sampling designs such as applied in the German NFI, which has not yet been the case for frequently used model-dependent estimators. Secondly, the design-based frame considerably relaxes requirements on the regression model which seemed appropriate facing severe quality restrictions in the auxiliary data of the study area. Thirdly, the estimators provide the asymptotically unbiased g-weight variance estimator which a) partially accounts for the design-dependency of the regression coefficients on the sample under the commonly applied \textit{internal model approach}, and b) is also robust to heteroscedasticity of model residuals.\par

The conduction of this study was divided into 3 work packages that each addressed major milestones towards the overall study objective. The work and results of these work packages are respectively presented in chapter \ref{chap:rpack}, chapter \ref{chap:regmod} and chapter \ref{chap:sae}. The following sections will give an introduction and some additional background information to each of these chapters.


% ---------------------- %
% ---------------------- %
\subsubsection{Work Package 1: Software implementation} % Chapter 2

% ----------------------------------------------
% # --- Chapter 2: Software implementation --- #

With respect to the study objective, work package 1 addressed the need of a robust and flexible software implementation of the design-based regression estimators that could handle large inventory data sets and process a large number of small area estimations at once. Whereas several of the estimators suggested by Mandallaz had been applied in simulations and real-world case studies \citep{mandallaz2013a, mandallaz2013b, mandallaz2013c, massey2014a, massey2015a, massey2015b}, there had yet not been an unified and consistent implementation of the estimators in the same software environment. The work on this study was thus taken as an opportunity to implement the full range of these regression estimators in the statistical software \proglang{R} \citep{R}. The implementation procedure comprised three steps in general: First, a comprehensive review of the regression estimators published by Mandallaz; second, the completion of yet missing estimators for three-phase small area estimation; and third, the actual implementation of the estimators in \proglang{R}. The latter seemed to be the software of choice, as it currently constitutes one of the most intensively used statistical software and also provides interfaces to data base systems in which inventory and geodata are commonly stored. A review of existing software for multi-stage and multi-phase estimation revealed that in comparison to official statistics, applications particularly suited for forest inventories have been rare. Exceptions are the \proglang{R} package \pkg{JoSAE} by \citet{josae2015} and the \pkg{maSAE} package by \citet{cullmann2016}. However, a more comprehensive software package covering a larger variety of sample designs and estimators - particularly in the design-based infinite population framework - had not yet been available. In order to address this lack between availability and recent interest in such methods, the software package has also been made freely available (\proglang{R} package \pkg{forestinventory}) and can be installed from the CRAN server (\url{https://CRAN.R-project.org/package=forestinventory}). Chapter \ref{chap:rpack} describes the implementation and the application of the two-phase and three-phase estimators in \proglang{R} and provides a comprehensive review of the design-based regression estimators for global and small area estimation described in \citet{mandallaz2008, mandallaz2013a, mandallaz2013c} and \citet{mandallaz2013b}. The availability of the software package in combination with its comprehensive documentation also had the objective to support the transparency and the reproducibility of the methods applied in this thesis.


%   - provides a review of all design-based estimators for global and small area estimation published by Mandallaz
%   - describes the implementation of the 2p/3p estimators in the statistical software R
%   - provides consistent framework to process large amount of data (plausibility checks, data intergrity, ...) 
%      objectives / requirements on software:
%      - provides calculation and comparison of the estimators based on the same input dataset
%      - provides summarized, standardized outputs for further analysis and visualization
%      - addresses the lack of available software for design-based estimations in forest inventories
%      - mention documentation: R-help, github, vignette
%      - end with: The availability of the software package in combination with its comprehensive documentation als had the objective to support the transparency and the reproducibility of the methods applied in this thesis.

%% for Discussion:
%   - development of a robust and flexible software application which allowed for the calculation of a comprehensive set of small area regression estimators
%   - the software is - at the moment - the most extensive open-source software for design-based estimation for forestry applications. The interest and need of such software was also indicated by
%     xxx download between the first and second release (maybe put this in Synthesis --> turned out that this software was needed and has widely been recognized ... / and feedback reveived from researchers as well as practioners).


% ---------------------- %
% ---------------------- %
\subsubsection{Work Package 2: Processing of auxiliary data and model building} % Chapter 3

%--------------------------------
% # --- Chapter 3: Modeling --- #

The objective of work package 2 was to find a suitable ordinary least square (OLS) regression model to be used as \textit{internal model} in the small area regression estimators. In order to apply the estimators to all management units in RLP, the regression model had to allow for predicting the standing timber volume of a German NFI sample plot at any location over the federal state forest area. This also imposed the restriction on the auxiliary data to be available at the federal state level. The explanatory variables were derived from country-wide airborne laser scanning (ALS) data, which were characterized by severe quality variations as well as time lags of up to 10 years between the ALS acquisitions and the terrestrial survey date. Consequently, an objective was to specifically address techniques to improve the performance of ordinary least square regression models under such restricting conditions. Additionally, the study also explored the use of tree species information derived from a country-wide tree species classification map as additional explanatory data. The integration of tree species information in timber volume prediction models has often been stated as some of the most promising but often missing and thus not well investigated information \citep{koch2010, white2016}. In this context, one yet existing gap of knowledge also concerned the effect of species misclassifications, i.e. errors in the explanatory variables, on the precision of the regression model. This question was addressed by proposing a calibration technique for removing a potential bias in the regression coefficients caused by such misclassifications. An additional challenge that further increased the complexity of the model selection procedure was the identification of optimal extraction areas (\textit{supports}) for the explanatory variables under varying plot sizes due to the angle count sampling technique applied in the German NFI. The overall question of the study was whether the frame of an OLS regression model provided enough flexibility to cope with the mentioned challenges in the data set. Besides these modeling-specific aspects, the work on this study comprised the integration and storage of both the terrestrial NFI data and the remote sensing data in a PostgreSQL database using a PostGIS extension. The latter allowed for a georelational storage and query of both data sources and provided fast computation of explanatory variables for large data sets.

% - some background info on data handling: - PostgreSQL-Server, PostGIS extension
% - Modeling the 'FI timber volume' based on available auxiliary data
% - basically constitutes the 'data preparation' for the sae-estimations 
% - consistent storage which provides data integrity, relations and flexbility in terms of queries ...
% - sanitize existing data to match requirements of estimators (geo-relational propoerties ...)
% - in this study only modeling on the plot level, an extended evaluation of the model on the cluster level is introduced in chapter 4
% - also addressing the question whether the frame of OLS regression models - which the estimators are restricted to if the g-weight variance is desired - provides enough flexibility to model the data

% ---------------------- %
% ---------------------- %
\subsubsection{Work Package 3: Small area estimation} % Chapter 4

% -------------------------------------
% # --- Chapter 4: SAE estimation --- #

Work package 3 comprised the actual application of the regression estimators for small area estimation and built upon a synthesis of the methods developed in work package 1 and 2. The aim of this study was to finally investigate which accuracies can be realized for timber volume estimation on small scale forest management units when using the German NFI data in the implemented small area regression estimators (work package 1). This first comprised an extension of the existing NFI sample grid in the study area (RLP) to a double-sampling cluster design, and the derivation of the explanatory variables used in the regression model (work package 2) at each sample location. Three types of design-based small area regression estimators were then applied to derive point and variance estimates of mean standing timber volume within 45 and 405 forest districts (\textit{Forst{\"a}mter} and \textit{Forstreviere}). The small area estimators considered were the \textit{pseudo-small}, \textit{extended pseudo-synthetic} and the \textit{pseudo-synthetic} design-based small area estimator for cluster sampling suggested by \citet{mandallaz2013a, mandallaz2013b}. An evaluation of the error distribution of these estimators on both small area levels served as a first means to quantify the estimation accuracies achievable under each estimator. The estimation results of the multi-phase estimators were also compared to the one-phase estimator for cluster sampling that exclusively uses the terrestrially observed data available within a small area unit in order to specify the gain in efficiency provided by the suggested double-sampling procedure. The results of our evaluations were subsequently used to discuss the potential of the suggested design-based regression estimators for future applications with respect to alternative auxiliary data and transferability to change estimation.

%\newpage
% --------------------------------- %
% --------------------------------- %
\subsection{Study 2: Mapping} % Chapter 5
\label{sec:study2}

% ###########################################################
% ## ===== general information on study 2 (Mapping) ====== ##

Forest attribute maps provide an area-wide overview of important information such as development stages, tree species or growing conditions and have thus always been of high interest for forest practitioners. For a long time, such maps were exclusively produced by hand and required expensive field visits and visual inspection of aerial photography. This amount of work also hampered a frequent updating of the maps. However, the production of maps (\textit{mapping}) covering large areas has lately been substantially supported by the availability of exhaustive remote sensing data in combination with modeling techniques \citep{brosofske2014}. Especially maps of predicted forest attributes in high resolution are considered to support the spatially precise allocation of management operations such as harvesting. An example is the use of rasterized timber volume prediction maps for the optimal allocation of cable roads in the frame of harvesting in steep slope mountainous terrains \citep{bont2012, bont2015}.\par

Despite the advantages of providing such high-resolution prediction maps, one should have in mind that the predictions are often made for considerably small spatial units (map pixels). In most cases, the map pixels match the extent of an inventory sample plot on which the prediction model has been calibrated. One can thus interpret mapping as an extreme case of \textit{small area estimation}. Design-based double-sampling estimators for small area estimation (study 1) provide closed-form variance formulas that allow to quantify the estimation precision for every small area unit individually by its estimation error and confidence interval. However, these concepts of quantifying the uncertainty cannot be transferred to mapping approaches in the particular case of a small area unit (i.e. map pixel) corresponding to the size of a sample plot. It is thus necessary that similar efforts for model building are invested in methods that reliably quantify the resulting map accuracies. A common way to characterize the map accuracy is to use criteria such as the coefficient of determination (R$^2$) or cross-validated root mean square error (RMSE), which rather address the overall prediction performance than the accuracy of individual predictions. For this reason, the specification of confidence intervals on pixel level has been stated as an important contribution to map accuracy assessment \citep{mcroberts2010a}.\par

In case of continuous response variables such as standing timber volume, the application of linear regression models allows for providing a confidence region for each prediction (i.e. pixel) based on the \textit{prediction interval} \citep[pp.136--139]{fahrmeir2013}. The objective of the study presented in Chapter \ref{chap:regmod} was to investigate an alternative approach of deriving pixel-wise confidence intervals by applying well-known concepts of accuracy assessment for categorical classification results \citep{congalton2008}. The core of the suggested approach was the definition of intervals within the range of terrestrial data and their respective model predictions, and subsequently calculate the \textit{user's accuracy} for each of those intervals. The calculated users' accuracies can then be regarded as the confidence levels for the chosen intervals. In this framework, an optimization algorithm (heuristic search method) is demonstrated that - given a pre-defined number of intervals - automatically identifies the interval boundaries with respect to the best possible classification accuracies. The motivation for the development of this method was twofold: first, to provide a map user with the possibility to evaluate the map detail, i.e. number of intervals/classes, in dependence of the realizable prediction accuracies; and second, to allow for identifying intervals of the response variable for which the map produces considerably high or low prediction accuracies. The applied procedures can also be downloaded as an \proglang{R} package from the GitHub development page \citep{github_classoptimr}.\par 

The suggested methods were applied in a mountainous study site in the canton of Grisons (Switzerland). Regional forest district inventory data were used in combination with data from an airborne laser scanning acquisition to produce a map of the standing timber volume on sample plot level. This map was subsequently evaluated by the developed accuracy assessment techniques. The setup of this study also addressed the overall question of the thesis, i.e. what prediction accuracies can be realized on small spatial scales when using forest inventory data that are only available in comparatively low sampling frequencies.


%% NOTES
% - Mapping of forest attributes has always been of huge interest for forest practicioners.
% - Especially pixelwise (rasterized) maps of predicted forest attributes are considered to allow for spatially precise allocation of management operations such as harvesting etc.
% - Examples for integration of timber volume prediction maps are e.g. the optimal allocation of cable roads for harvesting in steep slope mountaineuous terrain demonstrated by Bont
% - The production of maps covering large areas has been substantially supported by the availablility of exhaustive remote sensing data.
% - Despite the advantages of providing high-resolution pixelated predictions maps, one should have in mind that the predictions are made for considerably small spatial units that often match the extent of a sample plot on which the prediction model has been calibrated. 
% - We thus consider (emphasize that) mapping as an extreme case of small area estimation where the small area basically constitutes the area of a sample plot.
% - Classical double-sampling estimators for small area estimation (such as used in chapter 3) provide closed-form variance formulas that allow for quantifying the estimation precision for every small area unit individually by its estimation error or confidence interval. 
% - These concepts of uncertainty estimation can however not be transferred to mapping approaches, particularly in the extreme case of a small area (i.e. map pixel) corresponding to the size of a sample plot. 
% - It thus appears necessary that similar efforts than for model building should be invested in providing methods to quantify the resulting map accuracies. 
% - A common way to characterize the map accuracy is to use the R$^2$ (coef. of determ.) and the cross-validated RMSE (root mean squared error), which rather address the overall prediction performance than the accuracy for individual predictions. As deriving confidence intervals on pixel level has been stated as an important contribution to map accuracies assessment, one possibility in the frame of linear regression is to use the prediction intervals (Source: Fahrmeier pp. xx - yy).
%
% - Compared to overall accuracy metrics such as RMSE, we consider that individual 'unit/pixel-level' CIs provide a more detailed and realistic idea of the map accuracies and should be preferred.
% - Thus, the objective of our study was to contribute to the range of methods available to provide confidence intervals for each individual pixel of a prediction map.
% -  We investigated an alternative approach of deriving pixelwise confidence intervals by applying well-known concepts of accuracy assessment for categorical classification results (Congalton and Green). The basic idea of our approach is to define intervals in the set of terrestrial data and their respective model predictions which subsequently allow for calculating the user's accuracy for each of these intervals. These intervals and their respective users' accuracies can thus be treated as confidence intervals. In this framework, we demonstrate a new heuristic search method that identifies those intervals with respect to the best possible classification accuracies given a pre-defined resolution of the predictions (i.e. number of classes).

% for discussion:
% - Compared to overall accuracy metrics such as RMSE, we consider that individual 'unit/pixel-level' CIs provide a more detailed and realistic idea of the map accuracies and should be preferred.
% - for discussion: with regard to an integration of sae methods in operation forest management information systems, its has to be considered which auxiliary data are available on a long-term perspective.

