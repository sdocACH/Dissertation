%\setchapterpreamble[uc][0.8\textwidth]{%
%	\dictum[George Bernard Shaw]{%
%		``It is the mark of a truly intelligent person to be moved by statistics.''}\vskip20em}
	
%\setchapterpreamble[uc][0.8\textwidth]{%
%			\dictum[George Bernard Shaw]{%
%				``Science never solves a problem without creating ten more.''}\vskip20em}
			
\setchapterpreamble[uc][0.8\textwidth]{%
	\dictum[Han Solo]{%
		``She may not look like much, but she's got it where it counts.''}\vskip20em}			
	

	
\chapter{General Introduction}
\label{chap:intro}
\newpage

%------------------------------------------------------------------------------------------------%
% ---------------------------- History and state of the art------------------------------------- %
\section{History and state of the art}
\label{sec:intro:hist_soa}
\newpage

% end with: A conclusion from recent developments is that future forest planning methods will increasingly integrate information of auxiliary data such as remote sensing and geo data. 


%------------------------------------------------------------------------------------------------%
% ---------------------------------- Goals ----------------------------------------------------- %
\section{Thesis objective and structure}
\label{sec:intro:obj_and_struct}


% The objective of this thesis particularly was to contribute methodologies to the recent developments of operationally integrating information from auxiliary data 






%% NOTES:
%
% - main objectives:
%   o contribute methodologies for the combination of existing forest inventories with auxiliary data
%   o (thereby further) increasing the value of existing data from large scale forest inventories on small-scale spatial levels
%   o mainly concentrating on exploring the capabilities and performances of design-based multi-phase estimators in the service of small area estimation
%   o main question to be answered by case studies: --> what estimation / prediction accuracies can be realized using national forest inventory data on much smaller
%     spatial levels as the terrestrial inventory has been designed for?
%      --> investigated in two main studies:
%          1)*  Small area estimation of timber volume on forest district level using the German NFI data
%          2)** Mapping of timber volume on sample plot level using the Regional FDI of the canton of Grisons (CH)
%
% 
% -> 1) The first study constituted the major work of this thesis. 
% 
%
% 

% Ziele ableiten:
%
% - Ziel der Diss war insbesondere die Integrierung von SAE Methoden in bestehende Grossrauminventuren
%   - es stehen hier meist wenig Daten zur Verfügung zur Genauigkeitsabschätzung / Modellierung
% 
%   --> was ist das besondere daran: Es gibt bereits eine Menge an Publikationen / Untersuchungen, welche
%       solche Methoden in kleinen Untersuchungsgebieten testen. Es gibt auf der anderen Seite wenige Studien, 
%       welche 
%
%
% The overall goal of this thesis was 
%
% 2 main projects:
% - the first project, which basically occupied most of the time spent on this thesis, had the objective to develop a double-sampling estimation procedure for the German National Forest Inventory. It was of particular interest to evaluate whether a combination of already available remote sensing and geodata with the National Forest Inventory data can be a cost-saving alternative to increasing the terrestrial sampling sizes (implementation of a regional FDI). As the project was expected to provide a sound basis for further investment considerations into double sampling procedures, it was a prerequisite to gather the information over a sufficiently large area of Germany and its forests in order to allow for reliable conclusions. 















\newpage
%------------------------------------------------------------------------------------------------%
% ---------------------------------- Outline --------------------------------------------------- %
\section{Outline}
\label{sec:intro:outline}

In order to realize the project on small area estimations, three important working steps were identified at the beginning and defined as the major milestones of the project. The first milestone was the development of a robust and flexible software application which allowed for the calculation of a comprehensive set of small area regression estimators. This work required a 



% Notes here:
% 
















\begin{enumerate}
	\item Article I:
	\item Artilce II:
	\item Article III:
	\item Article IV:
\end{enumerate}


% Notes:
%
% o Paper I: 
%   - Summarizes all design-based estimators for small area estimation published by Mandallaz
%   - describes the implementation of the 2p/3p estimators in the statistical software R
%      - provides consistent framework to process large amount of data (plausibility checks, data intergrity, ...) 
%      - provides calculation and comparison of the estimators based on the same input dataset
%      - provides summarized, standardized outputs for further analysis and visualization
%      - addresses the lack of available software for design-based estimations in forest inventories
%        - facilitates their application, provides transparency and reproducability of the methods tested in this thesis
%
% o Paper II: 
%   - basically the 'data preparation' for the sae-estimations 
%   - tackles:  + processing of the data to make them suitable as input for SAE-software
%               + consistent storage which provides data integrity, relations and flexbility in terms of queries ...
%               + sanitize existing data to match requirements of estimators (geo-relational propoerties ...)
%   - objective: identify optimal processing of the data in order to achieve the best possible OLS model to use in the SAE software GIVEN the data
%
% o Paper III: 
%   - bringing together the work of Paper I and Paper II
%     -> application of SAE estimations
%   - objective: 1) which est.accuracies are already achievable with the current data (situation)?
%                2) give an idea what accuracies maybe possible with future data 
%
%
%
%
%
%
%
%
%
%
%















