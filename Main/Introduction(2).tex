%\setchapterpreamble[uc][0.8\textwidth]{%
%	\dictum[George Bernard Shaw]{%
%		``It is the mark of a truly intelligent person to be moved by statistics.''}\vskip20em}
	
%\setchapterpreamble[uc][0.8\textwidth]{%
%			\dictum[George Bernard Shaw]{%
%				``Science never solves a problem without creating ten more.''}\vskip20em}
			
\setchapterpreamble[uc][0.8\textwidth]{%
	\dictum[Han Solo]{%
		``She may not look like much, but she's got it where it counts.''}\vskip20em}			
	

	
\chapter{General Introduction}
\label{chap:intro}
\newpage

%------------------------------------------------------------------------------------------------%
% ---------------------------- History and state of the art------------------------------------- %
\section{History and state of the art}
\label{sec:intro:hist_soa}
\newpage

% end with: A conclusion from recent developments is that future forest planning methods will increasingly integrate information of auxiliary data such as remote sensing and geo data. 


%------------------------------------------------------------------------------------------------%
% ---------------------------------- Goals ----------------------------------------------------- %
\section{Thesis objective and structure}
\label{sec:intro:obj_and_struct}

The objective of this thesis was to contribute methodologies to the recent developments in combining existing forest inventory data from field surveys with auxiliary data, such as those derived from remote sensing data. The particular focus was thereby on investigating the potential increase in value of already existing large scale forest inventories to be used on small-scale management levels. The main question of this thesis to be addressed was, what estimation accuracies can be realized using forest inventory data on much smaller spatial levels as their sampling intensities have originally been designed for. This question was investigated in two main studies: The first study (study 1) constitutes the main part of this thesis and concentrated on exploring the capabilities and performances of design-based multi-phase regression estimators in the service of small area estimation. The second study (study 2) investigated a new approach for evaluating the estimation accuracies of forest attribute maps, which are considered to be a special case of small area estimation. The following sections will give a more detailed introduction to each of the studies.

%% NOTES:
%
% - main objectives:
%   o contribute to methodologies for the combination of existing forest inventories with auxiliary data
%   o (thereby further) increasing the value of existing data from large scale forest inventories on small-scale spatial levels
%   o main question to be answered by case studies: --> what estimation / prediction accuracies can be realized using national forest inventory data on much smaller
%     spatial levels as the terrestrial inventory has been designed for?
%      --> investigated in two main studies:
%          1)*  Design-based small area estimation of timber volume on forest district level using the German NFI data
%          2)** Mapping of timber volume on sample plot level using the Regional FDI of the canton of Grisons (CH)
%    o thesis is mainly concentrating on exploring the capabilities and performances of design-based multi-phase estimators in the service of small area estimation
%

% --------------------------------- %
% --------------------------------- %
\subsection{Study 1: Design-based Small Area Estimation}


% ############################################################
% ## ===== general information on study 1 (RLP SAEs) ====== ##







%
% -> 1) The first study constituted the major work of this thesis. The objective was to develop a double-sampling estimation procedure for the German National Forest Inventory. [Has been tested in other countires such as Norway, Switzerland, but no extensive study in Germany yet available]. As considered as a [first] study, it was thus of particular interest to evaluate whether a combination of already available remote sensing and geodata with the German National Forest Inventory data can be a cost-saving alternative to increasing the terrestrial sampling sizes (implementation of a regional FDI, which is under consideration in many federal German states). As the project was expected to provide a sound basis for further investment considerations into double sampling procedures, it was a prerequisite to gather the information over a sufficiently large area of Germany and its forests in order to allow for reliable conclusions. 
% 
%    ---> the work on this study was devided in 3 milestones/topics that are presented in chapter 2, 3 and 4. We will here provide information about the ... and the contribution of these chapters to the overall study objective, and also give some additional background information on the work of these chapters.


% ---------------------- %
% ---------------------- %
\subsubsection{Work Package 1} % Chapter 2



% ----------------------------------------------
% # --- Chapter 2: Software implementation --- #
%
%   - development of a robust and flexible software application which allowed for the calculation of a comprehensive set of small area regression estimators
%   - the software is - at the moment - the most extensive open-source software for design-based estimation for forestry applications. The interest and need of such software was also indicated by
%     xxx download between the first and second release (maybe put this in Synthesis --> turned out that this software was needed and has widely been recognized ... / and feedback reveived from researchers as well as practioners).
%
%   - provides a review of all design-based estimators for global and small area estimation published by Mandallaz
%   - describes the implementation of the 2p/3p estimators in the statistical software R
%   - provides consistent framework to process large amount of data (plausibility checks, data intergrity, ...) 
%      - provides calculation and comparison of the estimators based on the same input dataset
%      - provides summarized, standardized outputs for further analysis and visualization
%      - addresses the lack of available software for design-based estimations in forest inventories
%      - mention documentation: R-help, github, vignette
%      - supports the transparency and the reproducibility of the methods applied in this thesis!
%
%

% ---------------------- %
% ---------------------- %
\subsubsection{Work Package 2} % Chapter 3




%--------------------------------
% # --- Chapter 3: Modeling --- #
%
%   - Modeling the 'FI timber volume' based on available auxiliary data
%   - has also been addressed in recently publicated studies such as Maack et. al and Kirchhoefer et. al. As specialty / uniqueness of our study were a) the additional use of tree species information, b) the handling of substantially large time-lags between ALS data and NFI date, c) extensive analysis of optimal parameter settings under angle-count sampling
%
%   - basically the 'data preparation' for the sae-estimations 
%   - tackles:  + processing of the data to make them suitable as input for SAE-software
%               + consistent storage which provides data integrity, relations and flexbility in terms of queries ...
%               + sanitize existing data to match requirements of estimators (geo-relational propoerties ...)
%   - objective: identify optimal processing of the data in order to achieve the best possible OLS model to use in the SAE software GIVEN the data
%


% ---------------------- %
% ---------------------- %
\subsubsection{Work Package 3} % Chapter 4

% -------------------------------------
% # --- Chapter 4: SAE estimation --- #
%  - basically bringing the work of chapter 2 and chapter 3 together ...
%     -> application of SAE estimations
%   - objective: 1) which est.accuracies are already achievable with the current data (situation)?
%                2) give an idea what accuracies maybe possible with future data 
%
%
%
%
%

\newpage
% --------------------------------- %
% --------------------------------- %
\subsection{Study 2: Mapping} % Chapter 5

% ###########################################################
% ## ===== general information on study 2 (Mapping) ====== ##

Forest attribute maps provide an area-wide overview of important information such as development stages, tree species or growing conditions and have thus always been of high interest for forest practitioners. For a long time, such maps were produced by hand and required expensive field visits and visual inspection of aerial photography. This amount of work also hampered a frequent updating of the maps. However, the production of maps (\textit{mapping}) covering large areas has lately been substantially supported by the availability of exhaustive remote sensing data in combination with modeling techniques \citep{brosofske2014}. Especially maps of predicted forest attributes in high resolution are considered to support spatially precise allocation of management operations such as harvesting. An example is the use of rasterized timber volume prediction maps in the optimal allocation of cable roads for harvesting in steep slope mountainous terrains \citep{bont2012, bont2015}.\par

Despite the advantages of providing such high-resolution predictions maps, one should however have in mind that the predictions are often made for considerably small spatial units (map pixels). In most cases, the map pixels match the extent of an inventory sample plot on which the prediction model has been calibrated. One can thus interpret mapping as an extreme case of \textit{small area estimation}. Design-based double-sampling estimators for small area estimation (Study 1) provide closed-form variance formulas that allow for quantifying the estimation precision for every small area unit individually by its estimation error or confidence interval. However, these concepts of quantifying the uncertainty cannot be transferred to mapping approaches, particularly in the extreme case of a small area unit (i.e. map pixel) corresponding to the size of a sample plot. It is thus necessary that similar efforts than for model building are invested in methods that reliably quantify the resulting map accuracies. A common way to characterize the map accuracy is to use metrics such as the coefficient of determination (R$^2$) or cross-validated root mean square error (RMSE), which rather address the overall prediction performance than the accuracy of individual predictions. For this reason, the specification of confidence intervals on pixel level has been stated as an important contribution to map accuracy assessment \citep{mcroberts2010a}.\par

In case of continuous response variables such as standing timber volume, the application of linear regression models allows for providing a confidence region for each prediction (i.e. pixel) based on the \textit{prediction interval} \citep[pp.136--139]{fahrmeir2013}. The objective of the study presented in Chapter \ref{chap:regmod} was to investigate an alternative approach of deriving pixel-wise confidence intervals by applying well-known concepts of accuracy assessment for categorical classification results \citep{congalton2008}. The core of our approach was to define intervals within the range of terrestrial data and their respective model predictions, and subsequently calculate the \textit{user's accuracy} for each of those intervals. The calculated users' accuracies can then be regarded as the confidence levels for the chosen intervals. In this framework, we demonstrate an optimization algorithm (heuristic search method) that - given a pre-defined number of intervals - automatically identifies the interval boundaries with respect to the best possible classification accuracies. The motivation for the development of this method was twofold: 1) to provide a map user with the possibility to evaluate the map detail, i.e. number of intervals/classes, in dependence of the realizable prediction accuracies. 2) to allow for identifying intervals of the response variable for which the map produces considerably high or low prediction accuracies.\par 

The proposed methods were applied in a mountainous study site in the canton of Grisons (Switzerland). We used the regional forest district inventory data in combination with data from an airborne laser scanning acquisition to produce a map of the standing timber volume on sample plot level, on which we subsequently applied the developed accuracy assessment. The setup of this study also addressed the overall question of the thesis, i.e. what prediction accuracies can be realized on small spatial scales when using forest inventory data that are only available in comparatively low sampling frequencies.


%% NOTES
% - Mapping of forest attributes has always been of huge interest for forest practicioners.
% - Especially pixelwise (rasterized) maps of predicted forest attributes are considered to allow for spatially precise allocation of management operations such as harvesting etc.
% - Examples for integration of timber volume prediction maps are e.g. the optimal allocation of cable roads for harvesting in steep slope mountaineuous terrain demonstrated by Bont
% - The production of maps covering large areas has been substantially supported by the availablility of exhaustive remote sensing data.
% - Despite the advantages of providing high-resolution pixelated predictions maps, one should have in mind that the predictions are made for considerably small spatial units that often match the extent of a sample plot on which the prediction model has been calibrated. 
% - We thus consider (emphasize that) mapping as an extreme case of small area estimation where the small area basically constitutes the area of a sample plot.
% - Classical double-sampling estimators for small area estimation (such as used in chapter 3) provide closed-form variance formulas that allow for quantifying the estimation precision for every small area unit individually by its estimation error or confidence interval. 
% - These concepts of uncertainty estimation can however not be transferred to mapping approaches, particularly in the extreme case of a small area (i.e. map pixel) corresponding to the size of a sample plot. 
% - It thus appears necessary that similar efforts than for model building should be invested in providing methods to quantify the resulting map accuracies. 
% - A common way to characterize the map accuracy is to use the R$^2$ (coef. of determ.) and the cross-validated RMSE (root mean squared error), which rather address the overall prediction performance than the accuracy for individual predictions. As deriving confidence intervals on pixel level has been stated as an important contribution to map accuracies assessment, one possibility in the frame of linear regression is to use the prediction intervals (Source: Fahrmeier pp. xx - yy).
%
% - Compared to overall accuracy metrics such as RMSE, we consider that individual 'unit/pixel-level' CIs provide a more detailed and realistic idea of the map accuracies and should be preferred.
% - Thus, the objective of our study was to contribute to the range of methods available to provide confidence intervals for each individual pixel of a prediction map.
% -  We investigated an alternative approach of deriving pixelwise confidence intervals by applying well-known concepts of accuracy assessment for categorical classification results (Congalton and Green). The basic idea of our approach is to define intervals in the set of terrestrial data and their respective model predictions which subsequently allow for calculating the user's accuracy for each of these intervals. These intervals and their respective users' accuracies can thus be treated as confidence intervals. In this framework, we demonstrate a new heuristic search method that identifies those intervals with respect to the best possible classification accuracies given a pre-defined resolution of the predictions (i.e. number of classes).

% for discussion:
% - Compared to overall accuracy metrics such as RMSE, we consider that individual 'unit/pixel-level' CIs provide a more detailed and realistic idea of the map accuracies and should be preferred.
% - for discussion: with regard to an integration of sae methods in operation forest management information systems, its has to be considered which auxiliary data are available on a long-term perspective.

