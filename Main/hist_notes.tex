




%Since the 1950s, the demands on forest services have steadily been increasing. Beside the role of forests as wood supply, more diverse functions of forests became of importance, such as the recreational role, conservation of biodiversity, forest health and vitality (McRoberts et al. 2010) and also regulating and protective functions (Berger and Rey 2004; Frehner et al. 2005). As a result, today's forest inventories have to collect a huge amount of variables in order to provide quantitative data to report on each of these information needs. This has been supported by the development of a broad range of concepts and methods particularly suited to estimate forest inventory attributes \citep{gregoire2007, kohl2006, mandallaz2008, schreuder1993}. The large number of attributes to be collected has also further limited the number of sample plots than can be surveyed in a given amount of time, in particular under recent cost-restrictions. As a result, there has been an increasing need for alternative inventory methods that can maintain estimation precision at lower costs, i.e. fewer terrestrial sample sizes, or achieve higher estimation precision at identical costs \citep{vonluepke2013}. 
%
%A method which has been intensively used in this respect is multi-phase sampling. The core concept of such sampling procedures is to enlarge the sample size in order to gain higher estimation precision without enlarging the terrestrial sample size. This is done by using predictions of the target variable at additional sample locations where the terrestrial information has not been gathered. These predictions are generated by statistical models, commonly regression models, that use explanatory variables derived from auxiliary data. Regression estimators using this concept ususally use two samples of different sampling intensities (double- or two-phase sampling). According to \citet{schreuder1993}, double-sampling methods had already been developed by Cochran (1939), Neyman (1938) and Watson (1937) before they were extensively formulated in the forest inventory context by \citet{schreuder1993, kohl2006, gregoire2007, mandallaz2008}.
%
%Application of double-sampling methods in forestry became particularly interesting with the availablity of suitable auxiliary information. For a long time, this information was exclusively provided by aerial photographs. In Canada, aerial photographs had a long tradition and were already used in the 1930s in order to locate the inventory strips through merchantable forest stands. In Switzerland, Schmidt (1970) suggested a two-phase method where tree heights, stand densities, crown diameters and stem numbers are determined by a visual interpretation of aerial photos, and the terrestrial inventory is carried out only at a subsample. This method was then indeed implemented in the first Swiss NFI in 1983. Also Zöhrer (1980) in Germany already stated the use of areal photographs as a state-of-the-art of modern forest inventories. He described a comprehensive catalog of forest attributes to be derived from aerial photo interpretation, amongst others tree species, crown diameter, tree heigths as well as timber volume estimates on tree and stand level. He also suggested to use the resulting forest attribute maps for forest inventories in terms of building strata of different sampling frequencies.

% Major advancements in the incorporation of auxiliary data in forest inventory however emerged with a) the availability of large-scale remote sensing data technologies and b) increased computation capacities that allowed for automatically deriving information from that sources.

% Particularly the availability of small-footprint airborne laserscanning data 



% Recently, \citet{mandallaz2013c} and \citet{vonLuepke2012} extended double-sampling to triple-sampling and provided regression estimators that can use auxiliary information derived at two different sampling intensities.

% ...

%%This first section has the objective to outline important developments in forest inventories that subsequently led to the recent state of methods to which this thesis was intended to contribute. 
%The first known conductions of forest inventories date back to the 14th and 15th century. They looked quite different from today's inventories and exclusively comprised a visual inspection that was carried out by riding or walking through the forest. These inspections were a means to acquire a representative impression about the state of large forest areas as well as to determine spatial units for harvesting \citep{zoehrer1980}. The ability to estimate forest attributes by eye remained a crucial ability of forest practicioners for the next five centuries and is today still applied for stand inventories in many countries. Whereas these inspections constituted a response to increasing wood shortages, the first reference to sustainable forest management in literature is only found two centuries later in the book \textit{Silviculatura Oeconomica} written by \citet{carlowitz1713}. With the objective of a 'continuous and sustainable usage', Carlowitz suggested concepts that besides reforstation mostly targeted a more efficient use of wood.\par
%
%While the first use of sample plots to gather representative information date back to \citet{hartig1795}, major advancements of sample-based inventories came up at the beginning of the 20th century together with the development of statistical sampling methods. In North America, the first sample-based inventories (so-called \textit{timber cruises}) were conducted around 1930. The surveys were initially performed by visual assessment and later by a full census of all trees along systematically arranged lines. The idea of these so-called \textit{strip-sampling} inventories was to gather information only for a small percentage of the forest, which was then subsequently extrapolated to the entire forest area. The use of strips was at the same time also a means to produce forest attribute maps. An important further development of this sampling technique was the work of \citet{goodspeed1934} who advocated the collection of data within sample plots aligned with the strip lines (line-plot sampling) in substitution for a complete census within the stripes in order to improve the efficiency of the cruises. At the same time, \citet{langballe1938} demonstrated an early two-stage sampling approach in Canada in order to further reduce costs of a terrestrial cruise. They were confronted with considerable increases in survey costs triggered by stands becoming smaller, which required closer line spacing and more plots surveyed in the cruise. They thus proposed a method where field crews quickly counted the trees within each strip by eye, whereas they only performed time-consuming measurements at few plots within the strip. They subsequently used the relationship between the target variable (basal area) and the counted stem number to extrapolate the plot measurements to the entire strip area. This concept can be regarded as an early version of modern two-phase or two-stage estimation techniques.\par 
%
%Beside improving the efficiency, an issue that also became of high importance was to quantify the estimation error when sampling techniques were used. Solutions to this were developed in two mathematical frameworks that today constitute the two approaches for inference of estimates. One approach was the randomization of the plot locations, which was first recommended by \citet{hasel1938} in order to allow for a valid estimate of the sampling error under random sampling by applying the technique of variance analysis suggested by \citet{fisher1925}. For actual implementation, he suggested to devide the forest area into equally sized and shaped blocks and then randomly select a subset of these plots to be terrestrially surveyed. He thereby also described the so-called finite-population setup that builds the framework for a broad range of sampling methods \citep{schreuder1993}. At the same time, also other countries made huge advancement in the conduction of forest inventory methods. Around 1920, the first national, i.e. country-wide forest inventories (NFI) were conducted in the Nordic countries. Likewise the inventories in North America and Canada, also the Nordic NFIs were based on systematic strip sampling. This led to considerable contributions to variance estimation under \textit{systematic sampling}, particularly by Mat\'{e}rn (1947, 1960) who developed a variance estimator that relies on modeling spatial trends. The two theoretical frameworks of inference thus relied on either randomization of the sampling process or sampling from an underlying stochastic process and are today known as the \textit{design-based} and \textit{model-dependent} approach. A defining advancement in design-based survey sampling was made by the introduction of unequal propability sampling by \citet{hansen1943}, who showed that using inclusion probabilities proportional to the value of the target attribute could substantially increase estimation precision. An unbiased estimator for unequal probability sampling was contributed by \citet{horvitz1952}. The concept of \textit{probability proportional to size} (PPS) sampling is today implemented in most forest inventories by the use of concentric sample plots. A method which, even without intention at the time of invention, perfectly realizes PPS sampling is the angle-count sampling (ACS) technique introduced by Bitterlich in 1947 \citep{bitterlich1984}, and it was \citet{grosenbaugh1958} who later related ACS to the probabilistic sampling theory. A further important development was the reformulation of the design-based estimation frame within the \textit{infinite population approach} by \citet{mandallaz2008}. This approach provided a much better definition of the underlying population for forest inventories compared to design-based survey methods for finite populations as applied in official statistics.\par
%
%Since the 1950s, the demands on forest services have steadily been increasing. Beside the role of forests as wood supply, more diverse functions of forests became of importance, such as the recreational role, conservation of biodiversity, forest health and vitality (McRoberts et al. 2010) and also regulating and protective functions (Berger and Rey 2004; Frehner et al. 2005). As a result, today's forest inventories have to collect a huge amount of variables in order to provide quantitative data to report on each of these information needs. This has been supported by the development of a broad range of concepts and methods particularly suited to estimate forest inventory attributes \citep{gregoire2007, kohl2006, mandallaz2008, schreuder1993}. The large number of attributes to be collected has also further limited the number of sample plots than can be surveyed in a given amount of time, in particular under recent cost-restrictions. As a result, there has been an increasing need for alternative inventory methods that can maintain estimation precision at lower costs, i.e. fewer terrestrial sample sizes, or achieve higher estimation precision at identical costs \citep{vonluepke2013}. 
%
%A method which has been intensively used in this respect is multi-phase sampling. The core concept of such sampling procedures is to enlarge the sample size in order to gain higher estimation precision without enlarging the terrestrial sample size. This is done by using predictions of the target variable at additional sample locations where the terrestrial information has not been gathered. These predictions are generated by statistical models, commonly regression models, that use explanatory variables derived from auxiliary data. Regression estimators using this concept ususally use two samples of different sampling intensities (double- or two-phase sampling). According to \citet{schreuder1993}, double-sampling methods had already been developed by Cochran (1939), Neyman (1938) and Watson (1937) before they were extensively formulated in the forest inventory context by \citet{schreuder1993, kohl2006, gregoire2007, mandallaz2008}.
%
%Application of double-sampling methods in forestry became particularly interesting with the availablity of suitable auxiliary information. For a long time, this information was exclusively provided by aerial photographs. In Canada, aerial photographs had a long tradition and were already used in the 1930s in order to locate the inventory strips through merchantable forest stands. In Switzerland, Schmidt (1970) suggested a two-phase method where tree heights, stand densities, crown diameters and stem numbers are determined by a visual interpretation of aerial photos, and the terrestrial inventory is carried out only at a subsample. This method was then indeed implemented in the first Swiss NFI in 1983. Also Zöhrer (1980) in Germany already stated the use of areal photographs as a state-of-the-art of modern forest inventories. He described a comprehensive catalog of forest attributes to be derived from aerial photo interpretation, amongst others tree species, crown diameter, tree heigths as well as timber volume estimates on tree and stand level. He also suggested to use the resulting forest attribute maps for forest inventories in terms of building strata of different sampling frequencies.
