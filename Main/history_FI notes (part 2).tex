% --- second part ---

% Parts:
% ***
% part 1:
% - occurence of multi-phase estimation in officical statistics
% - first applications of mp methods in forest inventory --> triggered by the availablity of stereo photography and GIS between 1975 - 1990) --> era of visual photo interpreatation (still used)
% ****
% part 2:
% - the interest on mp methods has considerably increased with availability of rs data, particularly ALS data
% - review: overview of today's most important aux.infos (white et. al)
% ****
% part 3:
% - special application of small area estimation
% ****
% - end with: * A conclusion from recent developments is that future forest planning methods will increasingly integrate information of auxiliary data such as remote sensing and geo data. 
%             * The task is to develop and evaluate mp metods with regards to their operational use in large scale forest inventories   

% ------------------------------------------------------------------------------------- %
% ------------------------------------------------------------------------------------- %
% part 1:

% - introduction of multiphase inventory methods (field, when)
% - when first applied in forest inventory
% - first boost came with aerial phototgraphy
% - second boost came with increasing availability of remote sensing products


2010 McRoberts: 'Advances and emerging issues in national forest inventories'
-> "remotely sensed data have now been incorporated into operation forest inventories"
-> maps based on satellite images have been e.g. been used to optimize sample designs in Finland since 1992

































% ------------------------------------------------------------------------------------- %
% ------------------------------------------------------------------------------------- %
% part 2:
"most important and well investigated info comes from ALS data" ...

 - ALS data well suited to describe the vertical forest structure, single tree attributes, 

	- early ALS studies:
	
		*  1984 from Nelson et al.: 'Determining Forest Canopy Characteristics Using Airborne Laser Data'
		  -> conducted in Pennsylvania, findings: airborne laser scanner can be used to access the vertical structure of the forest canopy and is suitable to derive tree heights.
		  
		* 1986 from Maclean et al.: 'Gross-Merchantable Timber Volume Estimation Using an Airborne Lidar System'
		  -> they already tested ALS to be used for estimating gross-merchantable timber volume. They found ALS information to be a very significant and predictive variable to estimate timber volume
		     
		* 1988 from Nelson: 'Using airborne lasers to estimate forest canopy and stand characteristics'
		  -> already illustrated the reproducibility of ALS derived height, volume and biomass estimates under varying acquisition specifications, which has been supported by more recent studies. (Prerequisite for operational use).
		     
		* then, starting in 1995, ALS has been intensively tested in Norway:    
		     
			* 1997 from Naesset: 'Estimating Timber Volume of Forest Stands using Laser Scanner data'
			  -> first test of ALS in Norway
			  -> area-based approach tested and suggested to produce stand-wise volume estimates
			  -> study supported that mean canopy height can be accurately derived from ALS data
			  
			* 1997 from Naesset: 'Determination of mean tree height of forest stands using airborne laser scanner data'
			 -> investigated the estimation of the mean tree height of a forest stand, they found that ALS-based estimation
			    slightly underestimates the ground-truth. Study suggests that ALS mean tree height can be used instead of
			    derived by manual measurements from aerial photographs.			
			
			* 2002 from Naesset: 'Predicting forest stand characteristics with airborne scanning laser using a practical two-stage procedure and field data'
			  -> illustrates the area-based method, which is basically a two-phase synthetic small area estimation where the individual grid cell predictions
			     are aggregated into a synthetic stand estimate
			  -> 3D point clouds used for area-based estimation
			  -> 
			  
			* from Naesset in Springer-book: 
			  -> since 2002, the application of ALS-assisted forest inventories has become a common practice in the Nordic countries (Norway, Sweden). Through frequent collaboration with commercial service 
			     providers, a market for offering ALS assisted inventories developed in Norway, Sweden, Finland, Estonia
			  -> Naesset addresses five key issues when aiming for the introduction of ALS-based (or in general mp) inventories methods for operational use. One of them are full-scale demonstrations 
			     (i.e. large-scale applications) in the framework of 'operational projects'. 
			     The others issues are 1) proper and extensive scientific documentation, acceptance in the scientific, and communication to practitioners; 2) showing the utility and economical benefit, ...


    -  other sources: 
        * Forestry Applications of Airborne Laser Scanning 2014 (Maltamo, Naesset, Vauhkonen)
        * Koch 2010: Status and future of laser scanning ...
        * McRoberts 2010: Advances and emerging issues in national forest inevntories
        
% ------------------------------------------------------------------------------------- %       
% ------------------------------------------------------------------------------------- %       
% part 3:
        
        
        
        
        
        
        
        
        
        
        
        
        
        
        