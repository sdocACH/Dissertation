\chapter*{Abstract}
\label{chap:Summary}
\addcontentsline{toc}{chapter}{Abstract}
Modern utilization techniques enable the substitution of fossil resources by renewable biological resources like wood from forestry. In this context, the bio-economy contributes to reducing the dependency upon fossil raw material and, at the same time, to the reduction of carbon dioxide emissions. The forest sector, as the second largest producer of renewable resources for the bio-economy, plays an important role in this context as the success of companies from the bio-based sector crucially depends on their raw material supply. The sustainably achievable wood potentials in a region are naturally limited. Rising demands for wood by the growing bio-based sector therefore probably intensify the competition on the local wood market. In times of rising demands for wood from forests, the questions arise \textit{''How to reliably predict the sustainably available wood potentials from forestry?''} and \textit{''How can companies from the bio-economy sector obtain information about the location, quantity and availability of their demanded resources?''} If bio-economy companies want to establish on the wood market, will need to know if their demands can be met with the reachable potentials. To answer these questions, I present differentiated applied statistical models to predict available wood potentials on different temporal and spatial scales. All models enable decision makers to predict available resources prior to harvesting. Predictions of available potentials are interesting mainly for two reasons. Firstly, objective calculations of usable wood from forest operations may uncover recently unused potentials. Furthermore, reliable and accurate predictions of the expectable wood volume from forest usage strengthens calculation of the entire resource supply chain. Those enhanced predictions are not only relevant for forest enterprises but for also for the entire forestry and wood cluster.

A descriptive analysis about the availability of woody biomass in the European beech-dominated central Germany builds the empirical basis of this thesis (Chapter \ref{chap:hzb}). Germany's most important region for the supply of European beech wood is analyzed with regard to its raw material situation. It is shown that the available wood amounts were almost entirely used in the time period between 2002 ans 2012. This reinforces the need for methods able to predict the available wood potentials reliably. Further wood potentials for the bio-economy may be uncovered.

Subsequently, three explicit statistical methods for the support of distinct decision problems in the biomass supply chain are presented.

Biomass functions and nutrient contents have basically two advantages for the supply of the bio-economy sector with biomass from forestry (Chapter \ref{chap:bm}). They can be used to evaluate the available wood potentials of forest stands fully and they can strengthen the prediction accuracy of raw material flows. The biomass potential of a forest can only be utilized to an extent that, in the long-term, won't deplete the supply of plant available nutrients in the forest ecosystem. Using easily measured input data, biomass functions allow for a reliable prediction of tree species- and tree fraction-specific single-tree biomasses. In combination with nutrient content data, the site specific level of forestry, where the amount of plant available nutrients in the ecosystem is essentially unchanged, can be assessed. Furthermore, they can easily be applied to predict biomass amounts in the biomass supply chain. Biomass functions and nutrient contents for the main tree species can be found in the literature. For other tree species, like sycamore or ash, however, there are only few and very specific studies available. The first presented methods for decision support are therefore biomass functions and nutrient contents for European beech, oak, ash and sycamore. It is shown in a case study that the usage of oak biomass functions for the biomass prediction of sycamore and ash, as it is practiced today, leads to a massive overestimation of the stand specific biomass. The share of species-rich deciduous forest stands, and thereby the importance of tree specific biomass functions, is increasing. The introduced models can help predicting the wood potential of those mixed deciduous forest stands. They thus enable decision makers to exploit the usable wood potential entirely.

The second statistical model enables prediction of the economically reasonable viable wood potential of European beech trees on a single tree level. As in the first example, the model can be used to predict usable wood potentials entirely to uncover recently unused potentials. The advanced predictions can improve the reliability of entire biomass supply chains. The wood potential of a tree basically consists of the stem wood volume as well as the economically viable wood volume in the crown (Chapter \ref{chap:beech_crowns}). Due to the high morphological variability of European beech crowns, taper models, which are nowadays often applied for wood volume prediction, are not satisfactory for predicting the economically viable wood volume arising from crowns. The second introduced method is a computer aided model, able to predict the economically viable wood volume arising from crowns of European beech trees. It is shown that the economically viable wood volume in the crown significantly depends on the morphological type of European beech crowns. The model requires very intensive and complicated morphological measurements of specific crown branches. It is therefore not usable in the framework of the practical forest inventory. To make the results nevertheless applicable for practitioners, the modeling results are used to develop a regression formula able to predict the economically viable wood volume in the crowns of European beech trees. 

Combined forest growth and yield si\-mu\-la\-tion-op\-ti\-mi\-za\-tion procedures are important planning tools in an international framework and have shown their potential to support short-term operational decisions, while simultaneously considering long-term issues and intrinsic strategical orientations of the forest owners. The third developed method is a si\-mu\-la\-tion-op\-ti\-mi\-za\-tion software that is able to optimize the monetary return from forest usage in time horizons up to 20 years under consideration of the given conditions and restrictions of the forest enterprise (Chapter \ref{chap:opt}). Via iterative forest development simulations with changing harvesting intensities, an optimized forest development is calculated. The \textit{Tree Growth Open Source Software} (TreeGrOSS) of the Northwest German Research Institute is used as simulation module. The si\-mu\-la\-tion-op\-ti\-mi\-za\-tion was developed to support the intermediate-term planning of forest enterprises and to enhance collaboration between the forestry and the bio-economy sector. It is shown in a case study that the si\-mu\-la\-tion-op\-ti\-mi\-za\-tion model can calculate the forest development with the highest monetary return under given properties and restrictions. Contracts with binding delivery amounts between forest enterprises and wood processing companies cause opportunity costs, if they force decision makers to deviate from the favorable forest development plans. The si\-mu\-la\-tion-op\-ti\-mi\-za\-tion can be used to calculate such opportunity costs of binding delivery contracts. Forest owners can use the model results to decide whether the benefits in intermediate-term planning, implied by the contracts, justify their opportunity costs. The model results can build an objective basis for negotiation of intermediate-term delivery contracts between forest enterprises and bio-economy companies.