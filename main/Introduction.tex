%\setchapterpreamble[uc][0.8\textwidth]{%
%	\dictum[George Bernard Shaw]{%
%		``It is the mark of a truly intelligent person to be moved by statistics.''}\vskip20em}
	
%\setchapterpreamble[uc][0.8\textwidth]{%
%			\dictum[George Bernard Shaw]{%
%				``Science never solves a problem without creating ten more.''}\vskip20em}
			
\setchapterpreamble[uc][0.8\textwidth]{%
	\dictum[Han Solo]{%
		``She may not look like much, but she's got it where it counts.''}\vskip20em}			
	

	
\chapter{General Introduction}
\label{chap:intro}
\newpage

%------------------------------------------------------------------------------------------------%
% ---------------------------- History and state of the art------------------------------------- %
\section{History and state of the art}
\label{sec:intro:hist_soa}
\newpage

%% -------------------------------------------------------------------- %
%% 1. What were the triggers of conducting forest inventories? 
%%    Since when have they been conducted? Since when regular national inventories (Switzerland, Germany, Norway, ....)
%     (sustainability, ...)






% Infos: 
% o AWF-wiki (http://wiki.awf.forst.uni-goettingen.de/wiki/index.php/Brief_history_of_forest_inventory)
%   - first forest inventories in Europe carried out in the 14th and 15th century. 
%   - triggered by mining activities and the idea of systainability (Carl von Carlowitz, 1712)
%   - early inventories were crude assessments and not to compare with todays inventories.
%   - In 19th century, forest inventories were an established component of forest management planning, but based mainly on visual estimation (estiamtion methods not available yet)
%   - mathematical estimation methods developed around 1900.
%   - first large area inventories in Sweden around 1840 on provincial level 
%   - begin of first national forest inventories in the Nordic countries Norway, Sweden, Finland in 1910.
%   - changes and progresses in forest inventory mainly triggered by developments in statistical sampling theory, remote sensing, computers, measurement devices.


%% -------------------------------------------------------------------- %
%% 2. What were the major progresses / major changes in forest inventories and how were they triggered?





%% -------------------------------------------------------------------- %
%% 3. What are recent developments in the field of forest inventory?
% - huge area-coverage by RS-data (not possible by field surveys)
% - RS-data provide direct, but most often indirect information about the target variables
% - use / explore modelling approaches to create the link between RS-data and exact field information (short review here)
% - these models are however error-prone
% ==> that is the point where "dealing with uncertainty" comes into play (this is most often not addressed in articles)








% -------------------------------------------------------------------- %
% 4. What are the challenges, derived by point 3 ?
% - how to consistently \textit{store} the large amount of data in a way to easily assess them for a variety of validation purposes?
% - how to process \textit{store} the large amount of data (software)
% - ...
%
%









% Notes:
% - Anfänge der Waldinventur: Wo? Warum, mit welcher Motivation --> Nachhaltigkeit, Inventur ist Planungsgrundlage
%
% - Wie bisher? Was ändert sich gerade?
%
% - Hier herausarbeiten: Trend zu kombinierten Stiproverfahren in der Waldinventur
%   - welche Länder planen das? wo wird es bereits angewendet?
%
% - Motivation: Einbindung von Inventurinfo in Decision Support Systeme, Genauigkeiten mit einbeziehen 
%               --> Ungenauigkeiten klein halten bedeuted grössere Planungssicherheit
%
%  - Gleichzeitig müssen Inventurverfahren effecktiver werden (Personalmangel, zu grosse Flächen für akzeptable Genauigkeiten)
%
%  - Kurz diskutieren bzw. einfach erwähnen: es gibt 2 generelle Arten von komb. Verfahren: model-dependent und design-based
%    - pro design-based: praktische Anwendbarkeit (z.B. in official statistics)
%    - pro md: flexiblere Modellierungsmöglichkeiten, höhere Genauigkeiten WENN Modell "stimmt"
%
%
%




%------------------------------------------------------------------------------------------------%
% ---------------------------------- Goals ----------------------------------------------------- %
\section{Goals}
\label{sec:intro:goal}
\newpage

A conclusion from recent developments is that future forest planning methods will increasingly integrate information of auxiliary data such as remote sensing and geo data. As reliable information from forest inventories are the backbone of sustainable forest planning, 

However, 

% Ziele ableiten:
%
% - Ziel der Diss war insbesondere die Integrierung von SAE Methoden in bestehende Grossrauminventuren
%   - es stehen hier meist wenig Daten zur Verfügung zur Genauigkeitsabschätzung / Modellierung
% 
%   --> was ist das besondere daran: Es gibt bereits eine Menge an Publikationen / Untersuchungen, welche
%       solche Methoden in kleinen Untersuchungsgebieten testen. Es gibt auf der anderen Seite wenige Studien, 
%       welche 
%
%
% The overall goal of this thesis was 
%
% 2 main projects:
% - the first project, which basically occupied most of the time spent on this thesis, had the objective to develop a double-sampling estimation procedure for the German National Forest Inventory. It was of particular interest to evaluate whether a combination of already available remote sensing and geodata with the National Forest Inventory data can be a cost-saving alternative to increasing the terrestrial sampling sizes (implementation of a regional FDI). As the project was expected to provide a sound basis for further investment considerations into double sampling procedures, it was a prerequisite to gather the information over a sufficiently large area of Germany and its forests in order to allow for reliable conclusions. 
















%------------------------------------------------------------------------------------------------%
% ---------------------------------- Outline --------------------------------------------------- %
\section{Outline}
\label{sec:intro:outline}

In order to realize the project on small area estimations, three important working steps were identified at the beginning and defined as the major milestones of the project. The first milestone was the development of a robust and flexible software application which allowed for the calculation of a comprehensive set of small area regression estimators. This work required a 



% Notes here:
% 
















\begin{enumerate}
	\item Article I:
	\item Artilce II:
	\item Article III:
	\item Article IV:
\end{enumerate}


% Notes:
%
% o Paper I: 
%   - Summarizes all design-based estimators for small area estimation published by Mandallaz
%   - describes the implementation of the 2p/3p estimators in the statistical software R
%      - provides consistent framework to process large amount of data (plausibility checks, data intergrity, ...) 
%      - provides calculation and comparison of the estimators based on the same input dataset
%      - provides summarized, standardized outputs for further analysis and visualization
%      - addresses the lack of available software for design-based estimations in forest inventories
%        - facilitates their application, provides transparency and reproducability of the methods tested in this thesis
%
% o Paper II: 
%   - basically the 'data preparation' for the sae-estimations 
%   - tackles:  + processing of the data to make them suitable as input for SAE-software
%               + consistent storage which provides data integrity, relations and flexbility in terms of queries ...
%               + sanitize existing data to match requirements of estimators (geo-relational propoerties ...)
%   - objective: identify optimal processing of the data in order to achieve the best possible OLS model to use in the SAE software GIVEN the data
%
% o Paper III: 
%   - bringing together the work of Paper I and Paper II
%     -> application of SAE estimations
%   - objective: 1) which est.accuracies are already achievable with the current data (situation)?
%                2) give an idea what accuracies maybe possible with future data 
%
%
%
%
%
%
%
%
%
%
%















