\chapter*{Summary}
\label{chap:Summary}
\addcontentsline{toc}{chapter}{Summary}

The objective of this cumulative thesis was to contribute methods for incorporating auxiliary information, such as remote sensing data, in existing forest inventories. The overall focus of these methods were to increase the value of existing large scale inventories, which are usually characterised by low sampling frequencies, on small-scale management levels. This research question was investigated in two main studies.\par

The first study presented in this thesis focused on the application of latest design-based regression estimators for small area estimation. The objective of the study was to develop a double-sampling procedure for the German National Forest Inventory (NFI) and evaluate whether the implied combination of remote sensing data and the terrestrial NFI data can provide sufficient estimation precision on two small-scale management levels. The approach was applied by the example of timber volume estimation within two forest district levels in the German federal state of Rhineland-Palatinate. The work on this study is presented in three subsequent chapters: the first chapter gives a comprehensive review of model-assisted design-based small area regression estimators and demonstrates their implementation in a statistical software package that was developed for the purpose of this study. The second chapter addresses challenges that occurred during the identification of a suitable timber-volume regression model to be used in the estimators, i.e. 1) heterogeneity of the remote sensing data due to quality variations and time gaps to the terrestrial survey, 2) derivation of explanatory variables under angle count sampling, and 3) incorporation of tree species map information. The third chapter illustrates the implementation of the small area estimation procedure and evaluate its potential by comparing the estimation precisions to those derived under one-phase sampling, i.e. exclusively using the terrestrial NFI data available within each management unit. The major results were the following: it was demonstrated that on both management levels, the suggested double-sampling procedure was able to substantially reduce the estimation error compared to the standard one-phase approach. Additionally, post-stratification according to variables that reflect quality variations or temporal asynchronicity in the auxiliary information turned out to be an effective means to improve the precision of OLS regression models. The additional stratification to tree species led to a further improvement of model performance and suggests the use of such information in forthcoming inventories.\par

The second study was carried out in a model-dependent framework and investigated an alternative approach of deriving pixel-wise confidence intervals for forest attribute maps. The core approach was to use available inventory data for map validation by applying concepts of accuracy assessment usually known from evaluation of categorical classification results. The definition of intervals within the range of map predictions and their respective reference data were used to calculate the user's accuracies as the probability for each interval to reflect the ground truth. In this framework, an optimization algorithm was developed to automatically identify intervals that lead to optimal classification schemes. The method was tested on a timber volume map for a mountainous study site in Switzerland using data from the regional forest inventory as validation data. The results from the study suggested that resulting class accuracies can turn out to be substantially lower than overall model accuracy metrics such as $R^2$ or root mean square error might suggest. The validation method thus provides an additional accuracy metric that can be easily calculated and improves the information about the accuracy provided by forest attribute maps.

