\chapter*{Abstract}
\label{chap:Abstract}
\addcontentsline{toc}{chapter}{Abstract}

% overall objective & research question:
The objective of this thesis was to contribute procedures to increase the value of large-scale forest inventory data by extending their applicability also to estimation on small-scale forest management levels. Two methods were closer investigated which both rely on combining the terrestrially gathered inventory data with remote sensing data that are available over the entire inventory area. Criteria to evaluate the methods with respect to their future operational use were the reliable specification of the precision associated to the estimates, the magnitude of estimation precision achievable on the respective management levels, and the suitability of the procedures for large-scale applications.\par

% db sae:
The first approach comprised the application of design-based small area regression estimators and was tested in a case study carried out in the German federal state of Rhineland-Palatinate. The objective of the study was to develop a double-sampling procedure for the German National Forest Inventory (NFI) and to evaluate whether the implied combination of remote sensing data and the terrestrial NFI data can provide sufficient estimation precision for timber volume estimation on two small-scale forest district levels. The work on this study is presented in three subsequent chapters: the first chapter gives a comprehensive review of design-based small area regression estimators and demonstrates their implementation in a statistical software package. The second chapter addresses three major challenges with respect to the identification of a suitable timber volume regression model to be used in the estimators. These challenges comprised the occurrence of time gaps between the terrestrial survey date and the acquisition of auxiliary data, the optimal derivation of explanatory variables under unknown spatial extent of a terrestrial sample location due to angle count sampling, and incorporation of tree species information derived from satellite image classification as additional predictor variable including the treatment of misclassifications. The third chapter illustrates the implementation of the small area estimation procedure and evaluates its potential by comparing the estimation precisions to those derived under one-phase sampling, i.e., exclusively using the terrestrial NFI data available within each management unit. The application of the design-based small area regression estimators led to an average reduction of the variance by 43\% and 25\% on the two forest district levels compared to the one-phase estimator, and yielded average estimation errors of 5\% and 11\% respectively. These results support that the design-based regression estimators show great potential for NFI data to be additionally used to provide estimates on small-scale management units. Increasing time gaps between terrestrial survey and the auxiliary data acquisition led to a pronounced residual inflation which, however, could be mitigated by including the acquisition date as a categorical predictor in the regression model. Additionally, the main tree species of a sample plot was revealed to be a powerful predictor that can improve prediction performance when combined with vegetation height information. By introducing a calibration technique it was also possible to neutralize a further residual inflation caused by the effect of tree species misclassifications on the regression model coefficients. Using categorical information in the regression model such as the main tree species of a sample location and the time lag between the terrestrial and the remote sensing acquisition extended the approach to double-sampling for regression within post-strata, which has been a well-known and effective means to reduce the variance of estimates.\par

% mapping:
The second approach comprised the application of model-dependent mapping where the target area, i.e., a map cell, corresponds to the extent of a sample plot and thus exclusively comprises one model prediction. As such maps provide predictions in very high spatial resolution, they support the location of forest management activities on very small scales. This imposes high requirements on the precision of the estimates. Focussing on prediction maps for continuous forest attributes, the objective was thus to investigate an additional means to address the precision for each map cell. This was approached by calculating the user's accuracy, as known from classification accuracy assessment techniques, for prediction value intervals using terrestrial inventory data located within the mapping area as reference data. In this framework, an optimization algorithm was developed to automatically divide the prediction range into intervals of minimal width and highest possible estimation precision while at the same time ensuring statistically sound properties of the underlying error matrix. The methods were tested on a timber volume map for a mountainous study site in Switzerland using data from the regional forest inventory as validation data. The results provided evidence that the prediction errors associated with specific value intervals are often considerably larger than expected concerning goodness-of-fit describing the overall prediction model performance. The suggested validation method thus provides an additional accuracy criterion that can be easily calculated and improves the information about the accuracy provided by forest attribute maps. The optimization method can additionally be used to identify prediction intervals that reflect the potentially varying performance of the underlying prediction model across the prediction range.