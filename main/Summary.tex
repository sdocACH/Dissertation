\chapter*{Summary}
\label{chap:Summary}
\addcontentsline{toc}{chapter}{Summary}

% overall objective & research question:
The objective of this thesis was to contribute methodologies for incorporating auxiliary information derived from remote sensing data in forest inventory procedures with focus on practical implementation and application. It was thereby of particular interest to explore if such methods enable the use of terrestrial data from large scale forest inventories for estimation on much smaller spatial levels while still providing sufficient estimation precision. In this research framework, two approaches for small area estimation were closer investigated.\par

% db sae:
The first approach comprised the application of design-based small area regression estimators and was tested in a case study carried out in the German federal state of Rhineland-Palatinate. The objective of the study was to develop a double-sampling procedure for the German National Forest Inventory (NFI) and evaluate whether the implied combination of remote sensing data and the terrestrial NFI data can provide sufficient estimation precision on two small-scale forest district levels. The approach was applied by the example of timber volume estimation within two forest district levels in the German federal state of Rhineland-Palatinate. The work on this study is presented in three subsequent chapters: the first chapter gives a comprehensive review of design-based small area regression estimators and demonstrates their implementation in a statistical software package that was developed for the purpose of this study. The second chapter addresses three major challenges that occurred during the identification of a suitable timber volume regression model to be used in the estimators. These challenges comprised the occurrence of time-gaps between the terrestrial survey date and the acquisition of auxiliary data, the optimal derivation of explanatory variables under unknown spatial extent of a terrestrial sample location due to angle count sampling, and incorporation of tree species information derived from satellite image classification as additional predictor variable including the treatment of misclassifications. The third chapter finally illustrates the implementation of the small area estimation procedure and evaluate its potential by comparing the estimation precisions to those derived under one-phase sampling, i.e. exclusively using the terrestrial NFI data available within each management unit. It was found that the application of the design-based small area regression estimators led to a reduction of the variance on the two forest district levels by 43\% and 25\% on average compared to the one-phase estimator. The results thus confirmed that the design-based regression estimators show great potential for NFI data to be additionally used for estimation on small scale management units. Increasing time-gaps between terrestrial survey and the auxiliary data acquisition led to a pronounced residual inflation which however could be mitigated by including the acquisition date as categorical predictor in the regression model. Additionally, the main tree species of a sample plot revealed to be a powerful predictor that can improve prediction performance when combined with vegetation height information. By introducing a calibration technique it was also possible to neutralize a further residual inflation caused by the effect of misclassifications in the tree species information on the regression model coefficients. Using categorical information such as quality and tree species in the regression model extents double-sampling for regression to post-stratification, which has been a well-known and effective means to reduce the variance of estimates.\par

% mapping:
The second approach comprised the application of model-dependent mapping where the target area, i.e. a map cell, corresponds to the extent of a sample plot and thus exclusively comprises one model prediction. Within the framework of predictions maps for continuous response variables, the objective was to investigate an alternative means of providing a confidence interval for each map cell by calculating the user's accuracy for prediction value intervals based on available terrestrial inventory data in the mapping area.
. 










%The second study was carried out in a model-dependent framework and investigated an alternative approach of deriving pixel-wise confidence intervals for forest attribute maps. The core approach was to use available inventory data for map validation by applying concepts of accuracy assessment usually known from evaluation of categorical classification results. The definition of intervals within the range of map predictions and their respective reference data were used to calculate the user's accuracies as the probability for each interval to reflect the ground truth. In this framework, an optimization algorithm was developed to automatically identify intervals that lead to optimal classification schemes. The method was tested on a timber volume map for a mountainous study site in Switzerland using data from the regional forest inventory as validation data. The results from the study suggested that resulting class accuracies can turn out to be substantially lower than overall model accuracy metrics such as $R^2$ or root mean square error might suggest. The validation method thus provides an additional accuracy metric that can be easily calculated and improves the information about the accuracy provided by forest attribute maps.
