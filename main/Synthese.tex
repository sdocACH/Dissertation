	
\chapter{Synthesis}
\label{chap:synth}
\newpage


%---------------------------------------------------------------------------------%
% ---------------------------- Main findings ------------------------------------ %

\section{Main findings}
\label{sec:synth:mainfind}

% Design-based SAE in RLP:
%  
% - the proposed design-based small area regression estimators show great potential for the German NFI data to be additionally used for estimation of forest attributes on forest district level. 
%   In a first case study presented in this thesis, 
%
% - auxiliary data: tree species indicated to be a valuable additional auxiliary variable when combined with vegetation height data to predict standing timber volume. When derived from satellite or
%   remote-sensing based classification, the effect of misclassifiction
%
% - unbiased estimators preferable because ...
%



\newpage

%---------------------------------------------------------------------------------%
% ---------------------------- Limitations and criticism ------------------------ %

\section{Limitations and criticism}
\label{sec:synth:limits}

% Design-based SAE in RLP:
%
% - time-lags between terrestrial survey and acquisition of the auxiliary data
%
% - angle-count sampling hampers the derivation of auxiliary data since the exact extent around a sample location in which information was gathered cannot be determined
%
% - unbiasedness comes to the price of providing a minimum number of terrestrial observations in a small area. Regarding the question about a sufficient number, the case study indicated
%   that sample sizes in a small area $\leq 5$ can result in large estimation errors when combined with moderate and poor model fits. [Additionally ...]. These considerations suggest a minimum sample
%   size of at least 6 observations per small area. However, this number is likely not met for ...\% of the FR districts and can surely not be provided on smaller spatial scales such as forest stands.
%   If estimations on these scales are still requested, it might be inevitable to increase the sampling frequency of the terrestrial inventory or apply model-dependent estimators. The disadvantage of
%   model-dependent estimators however ...
%
%   --> "How much?": - the case study indicated that sample sizes in a small area $\leq 5$ can result in large estimation errors when combined
%                      with moderate and poor model fits
%                    - theoretically: ...
%
% - design-based estimators in general: 
%  * errors in response not considered in the quantification of the estimation uncertainties (response considered error-free)
%  * design-based estimators rely on randomization of sample locations, but are applied to systematic grids. ...
%
%
%


\newpage

%---------------------------------------------------------------------------------%
% ---------------------------- Implications for future work --------------------- %

\section{Implications for future work}
\label{sec:synth:future}

% - use auxiliary information!
% - minimize time-lags between terrestrial survey and acquisition of the auxiliary data --> especially crucial for change estimation
% - 
%
%
%
%
%


















\newpage

