\begin{otherlanguage}{ngerman}
\chapter*{Zusammenfassung}
\label{chap:Zusammenfassung}
\addcontentsline{toc}{chapter}{Zusammenfassung}

Das Ziel dieser Dissertation war es, Methoden für die Einbindung von Hilfsinformation aus Fernerkundungsdaten in bestehende Waldinventuren beizutragen. Unter speziellem Fokus auf die praktischen Implementierung und Anwendung stand die Frage im Vordergrund, ob durch Kombination mit Hilfsinformationen terrestrische Daten aus grossräumigen Nationalinventuren auch für Schätzungen mit akzeptablen Genauigkeiten auf deutlich kleinräumigeren Ebenen verwendet werden können. Dies wurde im Rahmen zweier Ansätze für Kleingebietsschätzungen näher untersucht.\par

Der erste Ansatz bestand aus der Anwendung design-basierter Regressionsschätzer für Kleingebietsschätzungen und wurde in einer Studie im Bundesland Rheinland-Pfalz (Deutschland) getestet. Zunächst wurde das bestehende Netz der Nationalen Waldinventur zu einem zweiphasigen Stichprobendesign erweitert. Anschliessend wurden durch die Kombination mit landesweit vorhandenen Fernerkundungsdaten Schätzung der Holzvorräte auf Forstamt- und Forstrevierebene berechnet und die erreichten Schätzgenauigkeiten evaluiert. Die Ergebnisse der Studie werden in drei aufeinanderfolgenden Kapitel vorgestellt: Das erste Kapitel gibt einen umfassenden Überblick über design-basierte Regressionsschätzer mit speziellem Fokus auf Kleingebietschätzungen und erläutert deren Implementierung in ein statistisches Softwarepaket welches im Rahmen der Studie entwickelt wurde. Im zweiten Kapitel wird auf die drei grössten Herausforderungen bei der Regressionsmodellierung eingegangen. Zu diesen Herausforderungen zählten der zeitliche Versatz von terrestrischen Aufnahmen und Erhebung der Fernerkundunsgdaten, die optimale Ableitung erklärender Variablen unter unbekannter räumlicher Ausdehnung der terrestrisch durchgeführten Aufnahme (Winkelzählprobe), und die Verwendung von Baumarteninformation aus Satellitenbildklassifikation als zusätzliche erklärende Variable. Das dritte Kapitel illustriert die Durchführung der Kleingebietschätzungen als Synthese aus Kapitel 1 und 2, und präsentiert einen Vergleich der erzielten Schätzgenauigkeiten mit jenen, welche sich unter ausschliesslicher Benutzung der terrestrischen Inventurdaten (einphasige Schätzung) ergeben. Es zeigte sich, dass durch Anwendung der Kleingebietsschätzer die Varianz der einphasigen Schätzung im Mittel um 43\% und 25\% auf Forstamt- und Revierebene reduziert werden konnte, was zu mittleren Schätzfehlern von 5\% und 11\% führte. Damit untermauern diese Resultate das grosse Potential von design-basierten Kleingebietsschätzern für die Verwendung von Nationalinventurdaten auf kleinräumigen Managementebenen. Zudem zeigte sich, dass Zeitdifferenzen zwischen terrestrischen- und Fernerkundungsdaten zu einer Verschlechterung der Modellgenauigkeit führen. Dieser Effekt konnte jedoch durch die Verwendung des Aufnahmedatums als kategorielle Variable im Regressionsmodell deutlich verringert werden. Auch führte neben der Vegetationshöheninformation die zusätzliche Verwendung der geschätzten Hauptbaumart eines Stichprobenpunktes zu einer Verbesserung der Modellgenauigkeit. Durch die Anwendung eines Kalibrierungsmodelles war es zudem möglich, die negativen Effekte von Fehlklassifikationen der Baumarten auf das Regressionsmodell zu neutralisieren. Die Einbeziehung der kategoriellen erklärenden Variablen im Regressionsmodell erweiterte dabei den gewählten Ansatz zu Post-Stratifizierung, welches ein bekanntes und effizientes Mittel zur Varianzreduktion von Schätzungen darstellt.\par

Der zweite Ansatz Bestand aus der Anwendung eines modelabhängigen Mapping-Verfahrens, in welchem der Flächenbezug der Modellprognosen (Pixel der Vorhersagekarte) der räumlichen Ausdehnung eines Stichprobenkreises entspricht. Damit solche räumlich hochaufgelösten Prognosekarten zur Lokation von forstlichen Eingriffen auf kleinräumigsten Ebenen benutzt werden können, ist der Anspruch an ihre Genauigkeit hoch. Daher war es Ziel, für Prognosekarten kontinuierlicher Waldattribute eine zusätzliche Methode zur Bereitstellung der Genauigkeit pro Kartenpixel zu testen. Dies wurde durch die Berechnung der Nutzergenauigkeit, welche für die Genauigkeitsanalyse von Klassifikationen angewendet wird, für definierte Prognoseintervalle realisiert. In diesem Rahmen wurde auch ein Optimierungsalgorithmus getestet, welcher die Spannweite der Prognosen automatisch in kleinstmögliche Intervalle mit grösstmöglichen Nutzergenauigkeiten einteilt. Die Methoden wurden in einem subalpinen Testgebiet in der Schweiz am Beispiel einer Holzvorratskarte getestet. Daten der kantonalen Waldinventur dienten dabei als Referenzdaten um die Genauigkeit der Prognosekarte zu validieren. Die Ergebnisse deuten an, dass die Vorhersagefehler einzelner Wertebereiche oft deutlich höher sind als auf Basis von generalisierenden Metriken der Modellgenauigkeit vermutet werden kann. Die vorgeschlagene Validierungsmethode bietet daher eine zusätzliche Genauigkeitsmetrik, welche einfach berechnet werden kann und die Information bezüglich der Kartengenauigkeit verbessert. Die Optimierungsmethode kann dazu benutzt werden, eine Prognosekarte in Wertebereiche variabler Länge einzuteilen, welche die zugrundeliegende Modellgenauigkeit reflektieren.

\end{otherlanguage}



